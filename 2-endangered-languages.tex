\section{Low Resource Languages: An Overview}\label{sec:endlang}
% The State of endangered languages and computational linguistics

In this section, I will outline the state of endangered languages. %Specifically, I will talk about how many languages there are in the world, using Ethnologue.

\subsection{Definitions}

Before going further, it makes sense to define what the terms \emph{endangered}, \emph{minority}, \emph{low-} and \emph{under-resourced}, and other terms like \emph{threatened} languages mean. Ultimately, they refer as a whole to languages which are in peril in some way. However, there have slightly different meanings in different contexts, and according to the scale and metric applied.

In this section, I will define these terms: endangered, minority, low-resource, under-resourced, threatened, moribund, dormant, extinct, computer (such as JavaScript), revitalised, and constructed languages. This will help inform why I've chosen to focus on \emph{low-resource} languages (LRLs), and specifically low-resource natural languages with living populations.

% \subsubsection{Endangered Languages}
%
% % From https://github.com/RichardLitt/endangered-languages
% \emph{Endangered languages} are human languages that are in danger of extinction. This is most often meant to mean that they have been identified as likely to lose all first language speakers in the near future, and thus cease functioning as a language with a population. %quote actual definition from somewhere else here
% When first speakers die, it is almost inevitable that the language as a whole will also die. However, at times second language speakers - those who learned it later in life, but not as their first language from birth - can keep a language alive in some form. As well, written records can be used to revive a language - albeit in a new form, as intricacies of the grammar and vocabulary are inevitably lost when the chain of transmission is broken. An example here would be Manx, which lost all of its first language speakers, but has a considerable population today of second-language learners who are in turn teaching first language classes to children, in the hopes that they will pass the language along to their children.
%
% Some languages lose all of their speakers, but may exist in another form. An example is Latin, which (excluding small communities of humanities scholars, amateurs, and priests) is almost exclusively used today only in written form, and normally read and not written.
%
% \subsubsection{Minority languages}
%
% Minority languages are are spoken by a stable, but small, population (for example, Maltese or Hawai'ian); and low- or under-resourced languages, which are spoken by a significant population but under-represented on the web (for instance, Quechua). These languages share certain characteristics in common; the most pertinent is sparse data and a lack of resources, ranging from spell-checkers to grammars to machine translation corpora. Other under-resourced languages that do not fall under this list include constructed languages (for instance, Klingon or Na'vi), computer languages (for instance, Javascript or Lua), and extinct languages that are so sparse as to be rendered computationally irrelevant for most purposes (for instance, Tocharian).
%
% \subsubsection{Low and Under Resourced Languages}
%
% \subsubsection{Other terms}
%
% - Computer languages
% - Constructed languages
% - Extinct languages
% - Moribund languages
% - Sleeping languages

\subsection{Metrics}

There are various metrics would can be used to assess language health. In this section, I'll explain these metrics in detail, focusing on the UNESCO, GIDS, EGIDS, and LEI measurements, as suggested by \citet{yang2017toward}.

% TODO Alexis: would be interesting to also discuss how these metrics might apply for the set of languages you're interested in (as defined in 2.1)

% Alexis: yes, in fact I think it's rather a different enterprise than defining language endangerment -- you may want to explicitly address the interaction between the two. what is your stance on the question? does lack of digital presence necessarily equate to language endangerment? (I think it's a very interesting question)

% > A variety of metrics have been proposed for assessing a language?s degree of endangerment ? the UNESCO system (2003), the Graded Intergenerational Disruption Scale (GIDS; Fishman 1991), the Extended Graded Intergenerational Disruption Scale (E-GIDS; Lewis & Simons 2010), and the Language Endangerment Index (LEI; Lee & Van Way 2016), among others. - yang_ogrady.pdf

% Describe UNESCO
% Describe GIDS
% Describe EGIDS
% Describe LEI


\subsection{Digital presence}

In this section, I am going to explain what digital presence is. This is more than just defining language endangerment - instead, this is about how do we quantify a language's existence digitally, either on the web or offline in archives.

Few of the metrics above take into account the level of digital literacy for a language. The possibility for a language to digitally ascend has been held up as a key component of judging a language's vitality by \citet{kornai2013digital}. %Reword wow
% Describe Kornai

I'll describe his assessment here, and explain why an alternative assessment would also be good. For instance, Wikipedia is, in my opinion, not a good judge of a language's health, as it is a closed ecosystem with diminishing returns for users who are bilingual.

% Talk about BLARK
% Describe a new metric using BLARK to judge the rate of a language's life


% Would be good to be able to cite something here. Otherwise it is original research.
% \begin{itemize}
% \item It is spoken by living fluent speakers, including second-language learners.
% \item It is spoken by living, first-language learners.
% \item It is productive in its morphology, growing in vocabulary, and not frozen in time.
% \item It is recorded in some form, including audio files.
% \item It has a writing system.
% \item It has a writing system that is used by modern speakers to record their own language.
% \item It has a writing system that can be used on a computer.
% \item The electronic writing system does not require excessive installation.
% \item All normal characters are available in Unicode.
% \item There is a growing corpus of written documents in the language.
% \item There are users who consistently use the language digitally.
% \item There is a formalized spelling system.
% \item There is a Bible translation. %Write about Wycliffe and SIL
% \item There are non-electronic documents.
% \item There is a dictionary.
% \item There is a machine-readable corpus.
% \item It is used on modern social media; Twitter, and so on.
% \item There is a Wikipedia entry.
% \item There are spellcheckers.
% \item There are syntactic tools.
% \item There are machine learning algorithms based on the language.
% \item There are speech-to-text or text-to-speech systems developed for the language.
% \end{itemize}
%
% To get a better idea of hose these metrics can be implemented, we can look at several different languages and how they use code to further language development.

% A language's digital presence is defined by: its ability to have tools which are easily accessible to users of the language, applications build on top of that tooling, large amounts of mono- and bilingual corpora, and an active amount of speakers (generally, in the millions). Generally, this also needs official backing in the form of an official status for a language, set at a national level. For instance, English is considered the {\it de facto} national language of United States, and the overwhelming majority of language users and computational software developed in the States is in English. In Europe, the EU mandates that citizens have the right to correspond with an EU institution (and receive a response back) in any of the 24 recognised official languages of the respective member states.\footnote{\href{https://europa.eu/european-union/topics/multilingualism\_en}{https://europa.eu/european-union/topics/multilingualism\_en}} This includes relatively small languages, such as Maltese, which has a significant web presence and tooling considering there are only just under half a million speakers. \footnote{\href{https://www.ethnologue.com/language/mlt}{https://www.ethnologue.com/language/mlt}} China is estimated to have more users of the internet than Japan, India, and the United States combined.\footnote{\href{http://chinapower.csis.org/web-connectedness/}{http://chinapower.csis.org/web-connectedness/}} But while the majority of these users will be speakers of Standard Mandarin, and not English; the operating systems and infrastructural backbone will still depend upon English-language originating software (although the user interface will be in Chinese).
%
% Neither official status nor population size necessarily means that a language will have resources or a digital presence, however. Romansh, although it is an official, provincial language of Switzerland, does not enjoy the same privileges as French or German, as there are only 40,000 speakers.\footnote{\href{https://www.ethnologue.com/language/roh}{https://www.ethnologue.com/language/roh}} Min Dong, although it has over nine million speakers \footnote{\href{https://www.ethnologue.com/language/cdo}{https://www.ethnologue.com/language/cdo}}, does not afford the same privileges as Mandarin, as it is not the official language of China. It is also not true that a language needs to have a large government presence or access to military-grade technology to have digital resources - a notable example would be Haitian Creole, spoken by over ten million speakers in one of the world's poorest countries. After the 2010 earthquake, Google, Microsoft, and Carnegie Mellon collaborated to make a machine translation system, an overwhelmingly difficult achievement.\cite{Lewis:2011:CMD:2132960.2133030}



\subsection{The current state of language diversity}

In this section, I am going to briefly go into detail about what diversity means for linguistics. This will be useful later for explaining how related languages can be used to bootstrap work in similar languages. For instance, Irish spell-checkers and constitutional corpora from the EU can be used by Scottish Gaelic speakers with some tweaks in order to further improve their own systems.



\subsection{Who makes resources for LRLs?}

Here, I will explain briefly who makes language resources for these languages. I'll explain what I see as the main groups doing this work: professional translators, educators, missionaries (of multiple faiths, but mostly Christian), academics and native technologists. I'll explain each stakeholder and their canonical perspectives.

% Alexis: what do you mean by "their canonical perspectives"? will this section address creation of digital resources only, or all resources? for that matter, what counts as a digital resource? does a collection of pdf scans of old books count? how about a pdf version of a text collection? etc....
% I don't immediately see how this fits into the rest of the thesis. also look at relevant publications from Jeff Good - one in particular on the ecology of language documentation: http://www.acsu.buffalo.edu/~jcgood/publications.html

% There are, of course, thousands of languages which could be used as examples of little to no resources, without recognition, speakers, or high literacy or computational use. The majority of languages do not have access to large financial bodies (whether military, governmental, or private) willing to invest in technical development, nor in developing large corpora which could be used to develop, bootstrap, and train computational tools. In such cases, any development work is normally undertaken by a few different groups; higher education departments or labs, native speakers with some digital ability, and missionaries, or, more usually, a combination of all three.
%
% These groups normally do not have access to large amounts of corpora, computational resources (such as servers, grids, or databanks), or, most importantly, manpower, time, and money. They are more often goal driven. For instance, Wycliffe %\TODO CITE
% translators are most interested in developing language literacy in order to translate the Bible into a language for first language speakers; they are often less interested in machine translation, Facebook groups, or developing literacy for schools unless it helps them towards this end. Native developers are normally more interested in developing apps or websites for locals, but they are often less interested in larger dictionaries, speech to text or text to speech resources, or use cases outside of their individual ken. Finally, academic staff are often forced to be myopic in scope in order to publish academically for their wider industry, and are less able to easily embark on decade-long timelines to developer singular resources. They are also often financially fragile, and depend on local investors or governmental budgets and stipends to fund their research, all of which often work on shorter timescales than missionaries or first language developers, who normally devote their entire lives to a particular language group. Finally, academics are often individual workers, working on a language family or small group at the expense of related languages, as they need are driven by the publish-or-perish academic market and do not usually have large amounts of funding for multilingual efforts.

\subsection{Language research funding}

Here, I'll go into more depth about funding, as we've outlined who works on LRLs and who would fund research, and why. This will further inform the basis for the work of the previous section. I'll talk about DARPA MT funding in the 20th century, as well as other efforts such as CLARIN.

