\section{The State of endangered languages}\label{sec:endlang}
% The State of endangered languages and computational linguistics
\subsection{Definining Endangered Languages}
\subsection{What are computational resources}


\subsubsection{The current state of language divesity}

\subsubsection{Digital presence}

A language's digital presence is defined by: its ability to have tools which are easily accessible to users of the language, applications build on top of that tooling, large amounts of mono- and bilingual corpora, and an active amount of speakers (generally, in the millions). Generally, this also needs official backing in the form of an official status for a language, set at a national level. For instance, English is considered the {\it de facto} national language of United States, and the overwhelming majority of language users and computational software developed in the States is in English. In Europe, the EU mandates that citizens have the right to correspond with an EU institution (and receive a response back) in any of the 24 recognized official languages of the respective member states.\footnote{\href{https://europa.eu/european-union/topics/multilingualism\_en}{https://europa.eu/european-union/topics/multilingualism\_en}} This includes relatively small languages, such as Maltese, which has a significant web presence and tooling considering there are only just under half a million speakers. \footnote{\href{https://www.ethnologue.com/language/mlt}{https://www.ethnologue.com/language/mlt}} China is estimated to have more users of the internet than Japan, India, and the United States combined.\footnote{\href{http://chinapower.csis.org/web-connectedness/}{http://chinapower.csis.org/web-connectedness/}} But while the majority of these users will be speakers of Standard Mandarin, and not English; the operating systems and infrastructural backbone will still depend upon English-language originating software (although the user interface will be in Chinese).

Neither official status nor population size necessarily means that a language will have resources or a digital presence, however. Romansh, although it is an official, provincial language of Switzerland, does not enjoy the same privileges as French or German, as there are only 40,000 speakers.\footnote{\href{https://www.ethnologue.com/language/roh}{https://www.ethnologue.com/language/roh}} Min Dong, although it has over nine million speakers \footnote{\href{https://www.ethnologue.com/language/cdo}{https://www.ethnologue.com/language/cdo}}, does not afford the same privileges as Mandarin, as it is not the official language of China. It is also not true that a language needs to have a large government presence or access to military-grade technology to have digital resources - a notable example would be Haitain Creole, spoken by over ten million speakers in one of the world's poorest countries. After the 2010 earthquake, Google, Microsoft, and Carnegie Mellon collaborated to make a machine translation system, an overwhelmingly difficult achievement.\cite{Lewis:2011:CMD:2132960.2133030}

\subsubsection{Interested parties for smaller languages}

There are, of course, thousands of languages which could be used as examples of little to no resources, without recognition, speakers, or high literacy or computational use. The majority of languages do not have access to large financial bodies (whether military, governmental, or private) willing to invest in technical development, nor in developing large corpora which could be used to develop, bootstrap, and train computational tools. In such cases, any development work is normally undertaken by a few different groups; higher education departments or labs, native speakers with some digital ability, and missionaries, or, more usually, a combination of all three.

These groups normally do not have access to large amounts of corpora, computational resources (such as servers, grids, or databanks), or, most importantly, manpower, time, and money. They are more often goal driven. For instance, Wycliffe %\TODO CITE
 translators are most interested in developing language literacy in order to translate the Bible into a language for first language speakers; they are often less interested in machine translation, Facebook groups, or developing literacy for schools unless it helps them towards this end. Native developers are normally more interested in developing apps or websites for locals, but they are often less interested in larger dictionaries, speech to text or text to speech resources, or use cases outside of their individual ken. Finally, academic staff are often forced to be myopic in scope in order to publish academically for their wider industry, and are less able to easily embark on decade-long timelines to developer singular resources. They are also often financially fragile, and depend on local investors or governmental budgets and stipends to fund their research, all of which often work on shorter timescales than missionaries or first language developers, who normally devote their entire lives to a particular language group. Finally, academics are often individual workers, working on a language family or small group at the expense of related languages, as they need are driven by the publish-or-perish academic market and do not usually have large amounts of funding for multilingual efforts.


\subsubsection{Language research funding}

\subsubsection{Open source as a solution}

\subsubsection{Current Open Source resources}



\subsection{Metrics}

There are various metrics would can be used to assess language health.

% > A variety of metrics have been proposed for assessing a language?s degree of endangerment ? the UNESCO system (2003), the Graded Intergenerational Disruption Scale (GIDS; Fishman 1991), the Extended Graded Intergenerational Dis- ruption Scale (E-GIDS; Lewis & Simons 2010), and the Language Endangerment Index (LEI; Lee & Van Way 2016), among others. - yang_ogrady.pdf

% Describe UNESCO
% Describe GIDS
% Descirbe EGIDS
% Describe LEI

However, few of these metrics take into account the level of digital literacy for a language. The possibility for a language to digitally ascend has been held up as a key component for the life of that language by Kornai. %Reword wow
% Describe Kornai


% Talk about BLARK
% Describe a new metric using BLARK to judge the rate of a language's life


% Would be good to be able to cite something here. Otherwise it is original research.
\begin{itemize}
\item It is spoken by living fluent speakers, including second-language learners.
\item It is spoken by living, first-language learners.
\item It is productive in it's morphology, growing in vocabulary, and not frozen in time.
\item It is recorded in some form, including audio files.
\item It has a writing system.
\item It has a writing system that is used by modern speakers to record their own language.
\item It has a writing system that can be used on a computer.
\item The electronic writing system does not require excessive installation.
\item All normal characters are available in Unicode.
\item There is a growing corpus of written documents in the language.
\item There are users who consistently use the language digitally.
\item There is a formalized spelling system.
\item There is a Bible translation. %Write about Wycliffe and SIL
\item There are non-electronic documents.
\item There is a dictionary.
\item There is a machine-readable corpus.
\item It is used on modern social media; Twitter, and so on.
\item There is a Wikipedia entry.
\item There are spellcheckers.
\item There are syntactic tools.
\item There are machine learning algorithms based on the language.
\item There are speech-to-text or text-to-speech systems developed for the language.
\end{itemize}

To get a better idea of hose these metrics can be implemented, we can look at several different languages and how they use code to further language development.

