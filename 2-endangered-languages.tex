% !TEX root = thesis.tex
\section{Low Resource Languages: An Overview}\label{sec:endlang}

In this section, I will outline the state of low resource languages. First I will define contrasting and distinct terms which are often used to these languages, which inform how one can approach a language. Then, I will talk about language demographics and metrics used to categorise languages as having low resources, before moving on to discuss digital presence as a term for understanding language endangerment today. Finally, I'll go into depth further about the current state of language diversity (both in research and demographically), and mention the various different groups who work on and fund low resource development, and how considering their impact influences a language's digital presence.

\subsection{Definitions}

Before going further, it makes sense to define what the terms \emph{endangered}, \emph{minority}, \emph{low} and \emph{under-resourced}, and other terms like \emph{threatened} mean when they refer to a language. Ultimately, they refer as a whole to languages which are in peril in some way. However, there have slightly different meanings in different contexts, and according to the scale and metric applied.

In this section, I will generally define these terms: {\it endangered}, {\it moribund},  {\it extinct}, {\it dormant}, {\it revitalised}, {\it historic} and  {\it constructed} languages; {\it minority}, {\it low-resource}, {\it under-resourced}, {\it incident} and {\it surprise} languages; and finally {\it computer} or {\it computational} languages. This will help inform why I've chosen to focus on low resource languages, and specifically low resource natural languages with living populations.

\subsubsection{Endangered, revitalised, and extinct languages}

\emph{Endangered} languages are human languages that are in danger of extinction. The term is borrowed from the scientific literature describing animals; just as there exists as very real possibility that one day there will be no more Australasian Bittern specimens in the wilds of Australia, it is also possible that one day there may be no living speakers of Guugu Yimithirr. The term is not complete analogous; we can still read Tocharian texts, but Tocharian is not considered to be a living language, but {\it extinct}, as there are no speakers who use it regularly (and who are not scholars of obscure languages).

Endangered languages are normally languages which have a high amount of speakers, and crucially are still teaching children the language. Children ensure that the language will live on to the next generation, and when this chain breaks, it is almost impossible to resurrect a language. A language would be endangered when it can be assumed that children will stop learning the language in the next hundred years (according to \citet{krauss92}). This can be difficult to judge, as the rate of deterioration can be high. For instance, Breton had over a million speakers in 1950, but today the numbers may be as low as 200,000. Its future is uncertain.

 \emph{Moribund} languages are languages which are critically endangered, in that there are no children currently learning the language and using it frequently, although there are speakers. Ainu is a good example, with roughly ten native speakers still living, all of whom are over 80 years old. % Cite
Haida has a similar amount of native speakers, but because of the amount of immersion programs, government-funded schools, and new venues for the language such as a motion picture filmed entirely in Haida with ethnically Haida actors who learned their lines from the elders, \footnote{\href{https://www.nytimes.com/2017/06/11/world/americas/reviving-a-lost-language-of-canada-through-film.html}{https://www.nytimes.com/2017/06/11/world/americas/reviving-a-lost-language-of-canada-through-film.html}} it is not considered moribund.

 \emph{Dormant} or {\it sleeping} languages are a stage beyond moribund languages. They have no living fluent speakers. This does not mean that the language is extinct. An example would be Mutsun, an Ohlone or Costanoan language formerly spoken near San Juan Bautista, California, whose last known fluent speaker Ascensi�n Sol�rsano passed away in 1930. However, in the late 90s, the Mutsun people (recognised formally as the Amah Mutsun Tribal Band)  began a revitalisation project using the extensive documentation left behind by linguists, anthropologists, and a Catholic mission priest, and now there are several conversational (albeit no fluent) speakers \citep{warner2007ethics}.

Often, dormant languages only come to attention when they are considered a {\it revitalised} language. As \citet{warner2007ethics} notes, "Daryl Baldwin did indeed teach himself his then-dormant ancestral language, Myaamia, and is now raising his children largely in the language \citep{hinton2001sleeping, leonard2004acquisition}." Before Baldwin's work, Myaamia would have been considered a dormant language. Another example would be Manx, which lost all of its native speakers (the last being Ned Maddrell, who died in 1974 \citep{wilson2008revitalization}), but retained a score of second language speakers until today, when there are now immersion programs for children and over a thousand speakers of the language \citep{clague2009manx}. Between 1974 and a vague point somewhere in the past couple of decades where a child could consider Manx as their first language, the language was dormant; now, however, it is revitalised.

The most famous example of a revitalised language is Hebrew, with a speaking population of over eight million,\footnote{\href{https://www.ethnologue.com/language/heb}{https://www.ethnologue.com/language/heb}} which was formerly a literary language until revitalisation efforts began as a result of the creation of the Israeli state in the early 20th century, where it is now an official language and not in a state of endangerment. Hebrew is a good example of why the often synonymous terms such as 'endangered' and 'revitalised' should be considered as differentiable.

While on the subject of Hebrew, it is worth mentioning that the initial efforts to revitalise it were often maligned by both Jewish communities and linguists, for a variety of reasons. First, the Jewish faith had traditionally viewed Hebrew as a holy tongue, and many religiously conservative Jews objected to the sacrilegious use of it for day-to-day matters, preferring Aramaic or Yiddish. Many also objected on the grounds that its use was connected to Zionism (why is well beyond the scope of this thesis). But most pertinently, linguists objected because they viewed revitalisation as an impossibility. If the language was dead, than it would be impossible to accurately bring it back, as literary texts are not sufficient at adequately capturing all of the intricacies of a language and how it is used. Clearly, with millions of first language speakers, this is no longer a valid point; these critics can now claim that modern Hebrew is an imperfect descendant of historical Hebrew, which remains extinct, and they are likely right to do so. Revitalisation is not always an ethically or logistically clear process.

This is especially true for {\it constructed} languages, which are {\it a priori} languages invented by a linguist or a community without a historical speaking community. These may be created to be logically resistant to ambiguity (such as Loglan or Lobjan), for a specific artistic purpose (such as Na'vi or Klingon, meant to be spoken by aliens in science fiction), for scientific study (such as those used by evolutionary linguists for language games with participants to discern how language might have evolved, or such as used in the ubiquitous Wug test by scholars of language acquisition) or for political aims (such as Esperanto or Ido). Some of these may end up with thousands of speakers, including native speakers, and a huge surplus of computational resources. Na'vi has a dictionary that has been translated using computational tooling into over a dozen languages, for instance, and morphological parser, spell checkers, and a Facebook translator. These languages are not normally considered as revitalised or dormant, but are instead mostly ignored by the scientific community altogether.

Heading back to natural languages, Latin would largely not be considered a revitalised language either, although there are immersion schools and some daily usage by the Catholic liturgy. These domains are specific and do not extend into normal life, on the whole. This doesn't mean it doesn't have some computational resources, however - the ATMs in the Vatican use Latin as a user interface language.\footnote{\href{https://gizmodo.com/5905595/the-atms-in-vatican-city-speak-latin}{https://gizmodo.com/5905595/the-atms-in-vatican-city-speak-latin}} Old Swedish, likewise, has some computational resources (admittedly, from a single research group that is humorously aware of the lack of general global interest in the field).\footnote{\href{https://spraakbanken.gu.se/swe/forskning/diabase}{https://spraakbanken.gu.se/swe/forskning/diabase}} Latin would normally be considered a {\it historic} language, like Ancient Greek or Old English. All of these languages, while extinct themselves, have direct descendants (the Romance languages, modern Greek, and English, respectively), but this is not always the case.

Gothic is considered {\it extinct} today, as it has no direct descendants, although it is still studied, and although there is a small community of writers who continue to use the language, and at least one publishing company which publishes modern work in Gothic\footnote{\href{https://wordhoardpress.com}{https://wordhoardpress.com}} (incidentally run by, of all people, me). Not all languages have sufficient texts to be revitalised or used today: Etruscan, Minoan, and Pictish are good examples.

One could argue that some languages may be considered dormant even if there are native speakers alive, if they do not speak the language. For instance, there are a few cases where a couple of speakers are left of a language, but they don't speak it to each other due to interpersonal differences. Most famously, there is the apocryphal story of Ayapeneco, where a global m\^eme ensued from an imagined feud between the last two speakers, to the point where Vodafone released a video claiming that they helped bring the men together to save the language (to the chagrin of actual linguists and anthropologists who had worked on the language for decades).\footnote{\href{http://stories.schwa-fire.com/who\_save\_ayapaneco\#chapter-113060}{http://stories.schwa-fire.com/who\_save\_ayapaneco\#chapter-113060}} This has actually happened elsewhere, such as with Nisenan \citep{snyder2004practice}. Another example might be Ishi, the last Yahi and a speaker of Yana, who explained that he had no name, because there was no other Yahi man to formally introduce him. Ishi means 'man' in Yana \citep{kroeber1973ishi}.

Such cases are extreme, and there will be exceptions to almost any of these categories. Even for living languages, questions of identification can be difficult. For instance, \cite{gilRiau} points to at least a dozen different interpretations of what Riau Indonesian might technically be. Defining language is beyond the scope of this thesis - however, I would be amiss not to mention this problem here.

\subsubsection{Minority, low and under resourced, and incident languages}

{\it Minority} languages are are spoken by a stable, but small, population (for example, Maltese or Hawai'ian); and low or under resourced languages, which are spoken by a significant population but underrepresented on the web (for instance, Quechua). These languages share certain characteristics in common; the most pertinent is sparse data and a lack of resources, ranging from spell-checkers to grammars to machine translation corpora. Other under-resourced languages that do not fall under this list include constructed languages (for instance, Klingon or Na'vi), computer languages (for instance, Javascript or Lua), and extinct languages that are so sparse as to be rendered computationally irrelevant for most purposes (for instance, Tocharian).

% \subsubsection{Low and Under Resourced Languages}
%
% \subsubsection{Other terms}
%
% - incident and surprise languages
% - Computer languages
% - Constructed languages
% - Extinct languages
% - Historic languages (Old Swedish, )
% - Moribund languages
% - Sleeping languages

\subsection{Metrics}

There are various metrics would can be used to assess language health. In this section, I'll explain these metrics in detail, focusing on the UNESCO, GIDS, EGIDS, and LEI measurements, as suggested by \citet{yang2017toward}.

% TODO Alexis: would be interesting to also discuss how these metrics might apply for the set of languages you're interested in (as defined in 2.1)

% Alexis: yes, in fact I think it's rather a different enterprise than defining language endangerment -- you may want to explicitly address the interaction between the two. what is your stance on the question? does lack of digital presence necessarily equate to language endangerment? (I think it's a very interesting question)

% > A variety of metrics have been proposed for assessing a language?s degree of endangerment ? the UNESCO system (2003), the Graded Intergenerational Disruption Scale (GIDS; Fishman 1991), the Extended Graded Intergenerational Disruption Scale (E-GIDS; Lewis & Simons 2010), and the Language Endangerment Index (LEI; Lee & Van Way 2016), among others. - yang_ogrady.pdf

% Describe UNESCO
% Describe GIDS
% Describe EGIDS
% Describe LEI


\subsection{Digital presence}



In this section, I am going to explain what digital presence is. This is more than just defining language endangerment - instead, this is about how do we quantify a language's existence digitally, either on the web or offline in archives.

Few of the metrics above take into account the level of digital literacy for a language. The possibility for a language to digitally ascend has been held up as a key component of judging a language's vitality by \citet{kornai2013digital}. %Reword wow
% Describe Kornai
%% Watch this video http://videolectures.net/metaforum2012_kornai_language/

%% Mention https://hlt.bme.hu/en/dld/

I'll describe his assessment here, and explain why an alternative assessment would also be good. For instance, Wikipedia is, in my opinion, not a good judge of a language's health, as it is a closed ecosystem with diminishing returns for users who are bilingual.

% Talk about BLARK
% Describe a new metric using BLARK to judge the rate of a language's life


% Would be good to be able to cite something here. Otherwise it is original research.
% \begin{itemize}
% \item It is spoken by living fluent speakers, including second-language learners.
% \item It is spoken by living, first-language learners.
% \item It is productive in its morphology, growing in vocabulary, and not frozen in time.
% \item It is recorded in some form, including audio files.
% \item It has a writing system.
% \item It has a writing system that is used by modern speakers to record their own language.
% \item It has a writing system that can be used on a computer.
% \item The electronic writing system does not require excessive installation.
% \item All normal characters are available in Unicode.
% \item There is a growing corpus of written documents in the language.
% \item There are users who consistently use the language digitally.
% \item There is a formalised spelling system.
% \item There is a Bible translation. %Write about Wycliffe and SIL
% \item There are non-electronic documents.
% \item There is a dictionary.
% \item There is a machine-readable corpus.
% \item It is used on modern social media; Twitter, and so on.
% \item There is a Wikipedia entry.
% \item There are spellcheckers.
% \item There are syntactic tools.
% \item There are machine learning algorithms based on the language.
% \item There are speech-to-text or text-to-speech systems developed for the language.
% \end{itemize}
%
% To get a better idea of hose these metrics can be implemented, we can look at several different languages and how they use code to further language development.

% A language's digital presence is defined by: its ability to have tools which are easily accessible to users of the language, applications build on top of that tooling, large amounts of mono- and bilingual corpora, and an active amount of speakers (generally, in the millions). Generally, this also needs official backing in the form of an official status for a language, set at a national level. For instance, English is considered the {\it de facto} national language of United States, and the overwhelming majority of language users and computational software developed in the States is in English. In Europe, the EU mandates that citizens have the right to correspond with an EU institution (and receive a response back) in any of the 24 recognised official languages of the respective member states.\footnote{\href{https://europa.eu/european-union/topics/multilingualism\_en}{https://europa.eu/european-union/topics/multilingualism\_en}} This includes relatively small languages, such as Maltese, which has a significant web presence and tooling considering there are only just under half a million speakers. \footnote{\href{https://www.ethnologue.com/language/mlt}{https://www.ethnologue.com/language/mlt}} China is estimated to have more users of the internet than Japan, India, and the United States combined.\footnote{\href{http://chinapower.csis.org/web-connectedness/}{http://chinapower.csis.org/web-connectedness/}} But while the majority of these users will be speakers of Standard Mandarin, and not English; the operating systems and infrastructural backbone will still depend upon English-language originating software (although the user interface will be in Chinese).
%
% Neither official status nor population size necessarily means that a language will have resources or a digital presence, however. Romansh, although it is an official, provincial language of Switzerland, does not enjoy the same privileges as French or German, as there are only 40,000 speakers.\footnote{\href{https://www.ethnologue.com/language/roh}{https://www.ethnologue.com/language/roh}} Min Dong, although it has over nine million speakers \footnote{\href{https://www.ethnologue.com/language/cdo}{https://www.ethnologue.com/language/cdo}}, does not afford the same privileges as Mandarin, as it is not the official language of China. It is also not true that a language needs to have a large government presence or access to military-grade technology to have digital resources - a notable example would be Haitian Creole, spoken by over ten million speakers in one of the world's poorest countries. After the 2010 earthquake, Google, Microsoft, and Carnegie Mellon collaborated to make a machine translation system, an overwhelmingly difficult achievement.\cite{Lewis:2011:CMD:2132960.2133030}



\subsection{The current state of language diversity}

In this section, I am going to briefly go into detail about what diversity means for linguistics. This will be useful later for explaining how related languages can be used to bootstrap work in similar languages. For instance, Irish spell-checkers and constitutional corpora from the EU can be used by Scottish Gaelic speakers with some tweaks in order to further improve their own systems.



\subsection{Who makes resources for LRLs?}

Here, I will explain briefly who makes language resources for these languages. I'll explain what I see as the main groups doing this work: professional translators, educators, missionaries (of multiple faiths, but mostly Christian), academics and native technologists. I'll explain each stakeholder and their canonical perspectives.

% Alexis: what do you mean by "their canonical perspectives"? will this section address creation of digital resources only, or all resources? for that matter, what counts as a digital resource? does a collection of pdf scans of old books count? how about a pdf version of a text collection? etc....
% I don't immediately see how this fits into the rest of the thesis. also look at relevant publications from Jeff Good - one in particular on the ecology of language documentation: http://www.acsu.buffalo.edu/~jcgood/publications.html

% There are, of course, thousands of languages which could be used as examples of little to no resources, without recognition, speakers, or high literacy or computational use. The majority of languages do not have access to large financial bodies (whether military, governmental, or private) willing to invest in technical development, nor in developing large corpora which could be used to develop, bootstrap, and train computational tools. In such cases, any development work is normally undertaken by a few different groups; higher education departments or labs, native speakers with some digital ability, and missionaries, or, more usually, a combination of all three.
%
% These groups normally do not have access to large amounts of corpora, computational resources (such as servers, grids, or databanks), or, most importantly, manpower, time, and money. They are more often goal driven. For instance, Wycliffe %\TODO CITE
% translators are most interested in developing language literacy in order to translate the Bible into a language for first language speakers; they are often less interested in machine translation, Facebook groups, or developing literacy for schools unless it helps them towards this end. Native developers are normally more interested in developing apps or websites for locals, but they are often less interested in larger dictionaries, speech to text or text to speech resources, or use cases outside of their individual ken. Finally, academic staff are often forced to be myopic in scope in order to publish academically for their wider industry, and are less able to easily embark on decade-long timelines to developer singular resources. They are also often financially fragile, and depend on local investors or governmental budgets and stipends to fund their research, all of which often work on shorter timescales than missionaries or first language developers, who normally devote their entire lives to a particular language group. Finally, academics are often individual workers, working on a language family or small group at the expense of related languages, as they need are driven by the publish-or-perish academic market and do not usually have large amounts of funding for multilingual efforts.

% SIGUL http://www.elra.info/en/sig/sigul/

%Include Elsnet here
%% From http://www.elsnet.org/index.html
% % The main objective of ELSNET-4 is to continue to advance Human Language Technologies by means of the operation of a network - ELSNET - encompassing a cross-section of private and public research centres actively engaged in research, development, integration or deployment. The network currently comprises 134 organisations in 27 countries, including the major players of Europe.
% % The network's collective knowledge and expertise form the basis of a number of actions and services aiming at improving Europe's competitive position and at increasing awareness and take-up of Human Language Technologies. Activities are designed to have an impact both within and outside the network and both during and after the lifetime of the project.
% %
% % ELSNET positions itself in an international context, both in relation to Northern America and the Far East, and to Central and Eastern European and Mediterranean countries.
% %
% % The work is focused around four main themes:
% %
% % Training
% % through summer schools and other training events and the operation of a Web directory of training opportunities and resources
% % Roadmapping
% % technological roadmaps for priority subareas of Human Language Technologies
% % Information
% % Dissemination
% % through a Web site for exchange of knowledge and expertise, a directory of experts, electronic mailing lists and a quarterly newsletter
% % Language Resources,
% % Standards and Evaluation
% % a roadmap for resources, internationalisation initiatives and the production of a map of the international resources landscape

% Also include ELDA http://www.elda.org/en/catalogues/language-resources-announcements/
%% The European Language Resources Association (ELRA) is a non-profit making organisation founded by the European Commission in 1995, with the mission of providing a clearing house for language resources and promoting Human Language Technologies (HLT).

\subsection{Language research funding}

Here, I'll go into more depth about funding, as we've outlined who works on LRLs and who would fund research, and why. This will further inform the basis for the work of the previous section. I'll talk about DARPA MT funding in the 20th century, as well as other efforts such as CLARIN.

