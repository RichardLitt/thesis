% !TEX root = thesis.tex
\section{Methods}
\label{sec:methods}

\subsection{Choosing a license}
\label{choosing-a-license}

I'll give some recommendations on a license, both for individuals and for larger companies. I am not a lawyer, so this will be short and tempered.

Work published without a license on a public site is not technically open source, either. When software is not licensed, it by default reverts (in the USA) to copyright where {\it all rights are reserved}, which is by definition not FLOSS. For this reason, it's important to add a license to code if it is in your purview to do so, and if you wish to follow the open source methodology. %This may belong elsewhere

\subsection{Choosing repositories}
\label{choosing-repositories}

I'll talk about my actual recommendations for storing code. I'll talk about how GitHub is a business, and its aims may not be aligned with researchers interested in long term archival, and similar concerns.


\subsection{Sharing code without a platform}
\label{subsec:sharing-code-without-a-platform}

I'll outline a plan for peer-to-peer resource sharing, using IPFS \citep{benet2014ipfs} and other related tech. I'll mention a case study involving local indigenous communities in Guyana using peer-to-peer to track illegally logging on their land, and explain how this system could also be used for language development.\footnote{\href{https://www.digital-democracy.org/}{https://www.digital-democracy.org/}} I'll discuss linguistic and scientific applications of using versioned, p2p, and distributed systems for storing both open source code related to low resource languages as well as language data.