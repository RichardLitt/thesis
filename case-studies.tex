% !TEX root = thesis.tex
\section{Case Studies}
\label{sec:case-studies}
\subsection{Scottish Gaelic}
\label{sec:gaelic}

Scottish Gaelic is a Celtic language spoken mainly in the United Kingdom, which UNESCO defines as {\it definitely endangered}.\footnote{\href{http://www.unesco.org/languages-atlas/en/atlasmap/language-iso-gla.html}{http://www.unesco.org/languages-atlas/en/atlasmap/language-iso-gla.html}} Gaelic - sometimes called Scots Gaelic, simply Gaelic, or the Gaelic - is a Goidelic or Q-Celtic langauge, along with Manx and Irish (also sometimes called Irish Gaelic, but here always referred to as Irish). This means that, while related to the Brythonic languages of Welsh, Cornish and Breton, it is different enough to not be able to benefit from the many resources available in Welsh, which, while endangered, has a much stronger academic interest and presence in the United Kingdom, with roughly half a million speakers.


A large corpus compiled by the An Crub\'ad\'an project is available online \footnote{\href{http://crubadan.org/languages/gd}{http://crubadan.org/languages/gd}} \citep{scannell2007crubadan}. % Mentioned directly in the text

ODIN has exactly 59 IGT entries for Scottish Gaelic.


As it is similar to Irish, it is a good example of how code from related languages can be used to bootstrap efforts to build code for its own language. I'll talk in depth about the language, its structure and grammar as related to code, its users and their use cases, and efforts to use code to make Scottish Gaelic digitally ascend.

%% Mention Ceitidh https://www.cereproc.com/en/CereProc_Gaelic_Synthetic_Voice_Ceitidh

%%  In 2001 the UK Government ratified the European Charter for Regional or Minority Languages (hereafter ECRML), recognising Ulster Scots and Irish in Northern Ireland, Welsh in Wales, Scots and Gaelic in Scotland, but not Cornish. Writing during the build-up to ratification, Dunbar (2000: 68) argued the case for Cornish: - Cornish, Sayers
\subsection{Naskapi}
\label{sec:naskapi}

% \subsubsection{Language Background}

Naskapi is a Cree language in the Algonquin family spoken in central Quebec \cite{MacKenzie-and-Jancewicz-1994}, which UNESCO defines as {\it vulnerable}.\footnote{\href{http://www.unesco.org/culture/languages-atlas/en/atlasmap/language-id-2354.html}{http://www.unesco.org/culture/languages-atlas/en/atlasmap/language-id-2354.html}} Virtually the entire population of around 800 Naskapi live within the reservation Kawawachikamach, around 10 miles from Schefferville, QC.

%In October 2017 I travelled to Schefferville and interviewed linguists working on a Naskapi bible, visited the school and talked to teachers at length about language efforts there, and talked to individuals around the town about their thoughts on the language and how it is used. I'll include a summary of Naskapi here, outlining current efforts and future possibilities for the language, and how open source code can help.

Schefferville is only accessible by train or plane, and contains another local tribe called the Innu (which has more than 17,000 members, scattered among Quebec and Labrador\footnote{https://en.wikipedia.org/wiki/Innu}), who live on their own reservation and who speak Montagnais or Innu-aimun, a related language. The two languages are similar, and the Naskapi youth are often diglossic in Montagnais (but the Innu are often not) \cite{MacKenzie-1980}.

The Naskapi speak English as a first or second language, while the Innu speak French (and some speak three or all four languages). They moved to Kawawachikamach in the 1960s, after initially being resettled in Schefferville in the early 1950s. Some of the elders still remember being a nomadic people who followed caribou and were raised in the bush. However, half of the population is under the age of 16, as the First Nations population is the largest growing population in Canada.\footnote{http://www12.statcan.gc.ca/census-recensement/2016/dp-pd/index-eng.cfm}

All of the Naskapi speak their own language regularly, in all contexts. In the schools, there are Naskapi-only classes held until Grade 8 \cite{llewellyn2017oral}. While there are a few social workers, teachers, and nurses who speak solely English, most jobs in Kawawachikamach are held by Naskapi. There has been a long tradition of missionaries, and almost all of the Naskapi are Protestant. At church, they use Montagnais hymnals and an Montagnais bible.

\subsubsection{Literacy developments}
In recent years, the Naskapi Development Council, which works with translators provided by the local tribal council (called the Band), has produced a Naskapi to English bilingual dictionary in three volumes \cite{MacKenzie-and-Jancewicz-1994}. This was produced by linguists from the Summer Institute of Linguistics, funded by Wycliffe Bible Translators.\footnote{\href{https://www.wycliffe.org/}{https://www.wycliffe.org/}}

% How was the document originally made?
Today, the SIL linguists are a team of six: two long term linguists, and two pairs of husband and wife pairs who are training how to work as bible translators in this community before moving on to working with other Cree communities in Canada. Naskapi does not have a complete bible. A new testament, started in the 70's, was recently published \cite{naskapi-new-testament}. Genesis, Exodus, and Psalms, have also been translated, and several children stories and books of oral legends from a an elder have been produced. The full-time translators are two people: a young woman in her mid-twenties, and an older gentleman of around 50 years of age. At times, elders also contribute to the bible translation effort by marking up their pre-publication drafts, which they then go over with the translators.

When there is a need to come up with a new term, the elders are consulted, and they agree on an appropriate translation. For instance, "grill" is translated as "metal-net". A grill is not a pre-existing word in Naskapi, but net is, and it is easy to imagine the metaphor of a grill on which you braise meat as being a metal net. However, these decisions are not replicated outside of the bible. Likewise, when there is a term which needs to be invented at the school, the teachers there decide on an appropriate term - for instance, for situations like Halloween, where "Frankenstein" may need to be translated into a local alternative. These decisions are largely one-off, although they may be used year to year, and informally recorded in their respective domains.

The linguists use the Fieldworks Language Explorer (FLEx) \footnote{\href{https://software.sil.org/fieldworks/}{https://software.sil.org/fieldworks/}} to document new linguistic terms. FLEx was developed by SIL International, and provides linguists with an out-of-the-box solution for recording linguistics terms using interlinear glossed text. It is also open source, and available on GitHub.\footnote{\href{https://github.com/sillsdev/FieldWorks}{https://github.com/sillsdev/FieldWorks}} Users can export as a PDF (among other file formats), or export words to an online interface known as Webonary.\footnote{\href{https://www.webonary.org/configuring-the-dictionary-in-flex/}{https://www.webonary.org/configuring-the-dictionary-in-flex/}}
This allows language workers to automatically create a useable, free dictionary for members of the community.

Naskapi uses the Inuit syllabics spelling system \cite{wals-141},
as well as two other roman-based systems with only minor differences. For instance, a macron, such as \^u is used in place of a double \emph{uu} to indicate vowel length. Computational writing using the syllabic system is possible by using Keyman,\footnote{\href{https://keyman.com/}{https://keyman.com/}} (free, open source software available on GitHub \footnote{\href{https://github.com/keymanapp}{https://github.com/keymanapp}})
which must be installed manually on a computer. It allows a user to type roman letters which are converted to the right syllabic phrase, and is forgiving for phonemic variants. For instance, "ju", "chu", "tchu" and so on might all be interpreted and replaced by the appropriate syllabic. %Add syllabic
%% Is this even possible in LaTeX
% Yes, it is, but it is fairly complicated. Use package `casyl`.

Currently, the school has a computer lab with over a dozen computers, but no in-house computer technician. One of the Wycliffe translators needed to visit the school to check on Keyman updates, and the students are not regularly trained in how to set up Keyman on their own, or how to set it up on their phones or other portable devices. While Facebook and other online platforms are increasingly popular, the majority of talking takes place in Naskapi written in local characters, or in English.

\subsubsection{Computational tools}
There are no spell checkers, word lists, or large corpora available digitally except for the dictionary. As well as the SIL-sponsored Webonary, there is also work done by atlas-ling.ca, which is a Canadian government-backed venture, originally cofounded by MacKenzie, who also worked on the Naskapi dictionary.\footnote{\href{http://atlas-ling.ca/}{http://atlas-ling.ca/}}
This website also has some options for looking at languages, but does not seem to be updated by local translators from the community. It is sourced from the previously published dictionary, which the SIL linguists have indicated is not up to date and has insufficient English to Naskapi translations. These are insufficient because of the nature of Naskapi; a root word is used with a slot system, and any word which mentions water is included under the English heading. This makes translating something as simple as "the mug is red" difficult, as you need to know to look for "red" as a root word, and then to find the appropriate example from which you can extrapolate the correct form for translation.

There is a potentially large corpus of spoken language in Naskapi from the local radio station, but this is not linguistically digested. There does not appear to be any adult-level secular written corpora which could be utilised to jump-start a corpus. The Band employs translators (who generally have other jobs - one this author interviewed was a band Councilman, one of four elected officials underneath the Chief) who may be able to provide bilingual texts in English, French, or Innu.

All told, computational work is exceedingly limited. There are some websites in Naskapi, which could be used to make a small corpus, but there are no currently active projects working on collecting corpora for the purpose of linguistic study, and neither is there an active academic community working on Naskapi outside of the SIL translators, who may occasionally publish a paper (or, of course, a dictionary or physical book).

While FLEx is open source, none of the linguists edit the code for it or use the codebase, depending on SIL International to keep the product up to date. Keyman is likewise not edited, although it is installed on local computers. There have been at least one Naskapi speaker who found and used a syllabic keyboard, but there has been no effort to standardise the syllabics in the schools or with other speakers, and the relevant code has not been shared in any official capacity by any party in the language community.

\subsubsection{Resources on the web}

ODIN has exactly one IGT entry for Naskapi, from \citep{richards2004syntax}.

The Naskapi community website, run by the Naskapi Nation of Kawawachikamach, has a webpage on installing Naskapi syllabics.\footnote{\href{http://www.naskapi.ca/en/Install-syllabics}{http://www.naskapi.ca/en/Install-syllabics}}

% TODO Add in Kornai's notes about Cree in kornai2013digital here
% TODO Add jancewicz2002applied and look at all papers which cite it
% TODO Mention http://www.language-archives.org/language/nsk