\section{Discussion}

Here, I want to drive home the point; how open source can help languages. Specifically, I will cover:

\subsection{Why isn't more code open?}

Finally, I'll go into a little detail on the question of why more hasn't been open sourced, and how to find open source resources.
% - Longevity of linguistic scholarship and work

\subsection{How does open source demonstrably help?}

I'll talk about use cases where open source has actually helped languages. This will include, for instance, NLTK case studies.

\section{Future Work}\label{sec:future-work}

Here, I'll talk about where to go next.

\subsection{Choosing a license}

I'll give some recommendations on a license, both for individuals and for larger companies. I am not a lawyer, so this will be short and tempered.

\subsection{Choosing repositories}

I'll talk about my actual recommendations for storing code. I'll talk about how GitHub is a business, and its aims may not be aligned with researchers interested in long term archival, and similar concerns.

% Longer term plans for open source repositories; GitHub is useful currently, but it also a business, and as such its aims may not be aligned with its users. I would like to talk about building a database of open source repositories on a secure, permanent, peer-to-peer network. This is something I am actively involved in professionally (I currently work at IPFS, which is building such a network). I would like to talk about linguistic and scientific applications of using versioned, p2p, and distributed systems for storing both open source code related to low resource languages as well as language data.

\subsection{Sharing code without a platform}

I'll outline a plan for peer-to-peer resource sharing, using IPFS \citep{benet2014ipfs} and other related tech. I'll mention a case study involving local indigenous communities in Guyana using peer-to-peer to track illegally logging on their land, and explain how this system could also be used for language development.\footnote{\href{https://www.digital-democracy.org/}{https://www.digital-democracy.org/}}

\subsection{Beyond Wikipedia and Ethnologue}

I'll talk about the shortcomings of both Wikipedia as a service, and Ethnologue as a provider of language data. Specifically, I want to draw attention to how Wikipedia treats its long-term contributors, and how Ethnologue charges exorbitant fees for using its data, and what we can do to improve this.

% \subsection{An Open Data Repository}

% I'll spec out the plans for an open data repository that could be used to share data.

% - Peer-to-peer solution for sharing code
%   - Stub out example
%   - Build a web searcher for automatically getting and sharing code
%     Further Work:
%   - Open source data repositories (touch on)
%   - Working with Ethnologue


% \subsection{Storage on a p2p network}
%
% Build a web-application tool for serving a decentralized data store for endangered language tools and data
%
% Example:
%
% I have already put a subset of repositories listed on endangered-languages into IPFS, a p2p resource for storing and disseminating data in a decentralized and persistent fashion.
%
% Process:
%
% 1. `cat` the endangered-languages README.md, then `grep` for `/.*(//github\.com/.*?/[a-zA-Z0-9-]*).*/` (all github.com repos).
% 2. Output list into separate file.
% 3. `awk` the first few repos, until a random divider, and clone the git repos: `awk '1;/kuromoji-server/{exit}' ../githublist.md | xargs -n1 git clone`
% 4. `ipfs add -r repos`
% 5. `ipfs pin add repos`



\section{Conclusion}\label{sec:conclusion}

Here I will conclude with some closing remarks.