% !TEX root = thesis.tex
\section{Open Source Code}
\label{sec:open-source}

% Changing tack, here I will talk about what \emph{open source} means. This is important - otherwise, this thesis is just a rehash of current existing computational work on LRLs.

\subsection{Defining {\it open source}}
\label{subsec:defining-open-source}

{\it Open Source} is a complex term which refers to any code, not just code related to computational linguistics. Here, I'll define what I mean by Open Source. This will largely inform the next section where I talk about its use for low resource languages.

At its core, {\it open source} refers to code which has a license which allows it to be available to freely inspect, use, or modify by anyone. It was introduced in 1998 by Linux programmers such as Eric Raymond, author of {\it The Cathedral and the Bazaar}\footnote{\href{http://www.catb.org/~esr/writings/cathedral-bazaar}{http://www.catb.org/~esr/writings/cathedral-bazaar}}\citep{raymond1999cathedral}; Linus Torvalds, author of the Linux kernel\footnote{\href{https://www.kernel.org/}{https://www.kernel.org/}} and Git\footnote{\href{https://git-scm.com/}{https://git-scm.com/}}; Richard Stallman, founder of the GNU project\footnote{\href{https://www.gnu.org/}{https://www.gnu.org/}} and the Free Software Foundation\footnote{\href{https://www.fsf.org/}{https://www.fsf.org/}}; and others in response to the Netscape browser's code being openly licensed and made available.

{\it Open source} is one of many terms which can be used to differentiate code which is either available or licensed permissively for re-use; other terms include {\it free} and {\it libre} software. There is no standard definition of open source that is universally accepted.

Nor will universal acceptance be forthcoming. The issue regarding reconciliation between open source, free software, and the rest of the terms stems largely from a difference of opinion between what constitutes open software, and what free and open means. An oft-used expression is "free as in beer"as opposed to "free as in speech", where the first is used for gratis software which has no monetary price set on it, as opposed to software which is written without restriction. The term {\it libre} is most often used for the second, to differentiate the two meanings in English. Occasionally, the acronym FLOSS is used in open source parlance to refer to Free Libre Open Source Software, which is both gratis and libre software.

For some adherents, software ought to be free (gratis), as it is a result of human labour and because opening it up without cost maximises the utility function of that code, and minimises duplicated effort. This idea contains within it the seed of the digital commons: like the commons in philosophical and economic literature, code can be viewed as a resource that belongs to humanity as a whole, and not the creators who initially fashioned it. In this sense, open source is a more of a philosophical theme than a technical term.

\begin{quote}
Open source is a development methodology; free software is a social movement. For the free software movement, free software is an ethical imperative, essential respect for the users' freedom. By contrast, the philosophy of open source considers issues in terms of how to make software  "better" - in a practical sense only. It says that nonfree software is an inferior solution to the practical problem at hand.\footnote{\href{https://www.gnu.org/philosophy/open-source-misses-the-point.html}{https://www.gnu.org/philosophy/open-source-misses-the-point.html}}
\signed Richard Stallman (Founder of GNU\/Linux)
\end{quote}

However, for the most part, open source isn't disambiguated as a term, because authority for this task is relegated to the license put on a piece of software, which determines the legality and potential use. Licenses determine the legal rights to sharing code. A piece of code which is taken from a proprietary server and published on the internet is not necessarily open source. In this instance, the code may have been illegally copied and shared, but it is not licensed for free usage. Under no definitions is this considered open source. Indeed, this touches upon issues of digital copytheft and piracy, which is a standard term used frequently in the media and in legal proceedings to attach a sense that copying code is the same as larceny or theft on the high seas. Avoiding the question of the validity of this viewpoint, it is important to focus on the license as the differentiating factor between code which has been released legally under an open definition or not. The term open source under most definitions doesn't pertain to ethical concerns about the software's usage, but rather simply refers to whether or not it is permissively licensed and available for users.

For that matter, work published without a license on a public repository site, such as on GitHub\footnote{\href{https://github.com}{https://github.com}}, is not technically open source, either. When software is not licensed, it by default reverts to copyright where{\it all rights are reserved}, which is by definition not FLOSS. For this reason, it's important to add a license to code if it is in your purview to do so, and if you wish to follow the open source methodology. There are many licenses which are considered to be open source, and there are several arbiters available which judge the validity of open source licensing. The Open Source Initiative (OSI) maintains a list of approved licenses on their website\footnote{\href{https://opensource.org/licenses}{https://opensource.org/licenses}}.

The OSI, whose founders were one of the original coiners of the term {\it open source}, has several parameters by which open source software can be judge as being 'open' or 'closed' (that is, proprietary, non-permissively licensed, non-reusable, limited in usage to a set amount of people, and so on). It may be useful to list these terms directly below, as they are instructive about how open source can be a nuanced term. These terms and their definitions are from the OSI's website \footnote{\href{https://opensource.org/osd}{https://opensource.org/osd}}, and are repeated below verbatim.

\begin{enumerate}
\item{Free Redistribution}. The license shall not restrict any party from selling or giving away the software as a component of an aggregate software distribution containing programs from several different sources. The license shall not require a royalty or other fee for such sale.
\item{Source Code}. The program must include source code, and must allow distribution in source code as well as compiled form. Where some form of a product is not distributed with source code, there must be a well-publicized means of obtaining the source code for no more than a reasonable reproduction cost, preferably downloading via the Internet without charge. The source code must be the preferred form in which a programmer would modify the program. Deliberately obfuscated source code is not allowed. Intermediate forms such as the output of a preprocessor or translator are not allowed.
\item{Derived Works}. The license must allow modifications and derived works, and must allow them to be distributed under the same terms as the license of the original software.
\item{Integrity of The Author's Source Code}.
  The license may restrict source-code from being distributed in modified form only if the license allows the distribution of "patch files" with the source code for the purpose of modifying the program at build time. The license must explicitly permit distribution of software built from modified source code. The license may require derived works to carry a different name or version number from the original software.
\item{No Discrimination Against Persons or Groups}.
  The license must not discriminate against any person or group of persons.
\item{No Discrimination Against Fields of Endeavor}.
  The license must not restrict anyone from making use of the program in a specific field of endeavor. For example, it may not restrict the program from being used in a business, or from being used for genetic research.
\item{Distribution of License}.
  The rights attached to the program must apply to all to whom the program is redistributed without the need for execution of an additional license by those parties.
\item{License Must Not Be Specific to a Product}.
  The rights attached to the program must not depend on the program's being part of a particular software distribution. If the program is extracted from that distribution and used or distributed within the terms of the program's license, all parties to whom the program is redistributed should have the same rights as those that are granted in conjunction with the original software distribution.
\item{License Must Not Restrict Other Software}.
  The license must not place restrictions on other software that is distributed along with the licensed software. For example, the license must not insist that all other programs distributed on the same medium must be open-source software.
\item{License Must Be Technology-Neutral}.
  No provision of the license may be predicated on any individual technology or style of interface.
\end{enumerate}

\subsection{Open source licenses}
\label{subsec:licenses}

These different terms and conditions are often conflated, and a legally-valid license which satisfies all of them is difficult to write on an {\it ad hoc} basis. For this reason most open source programming relies on using existing licenses, and copying them for specific projects. There are tools today to help make licensing more clear to na�ve users, such as \href{https://choosealicense.com}{choosealicense.com}, \href{https://tldrlegal.com}{tldrlegal.com}, and so on.

Some of the main licenses used in the wild are as follows:

\begin{itemize}
\item The MIT license, developed at MIT, is the most popular license on GitHub, the world's largest repository of code, used in over 40\% of the projects licensed there as of March 2015.\footnote{\href{https://blog.github.com/2015-03-09-open-source-license-usage-on-github-com/}{https://blog.github.com/2015-03-09-open-source-license-usage-on-github-com/}} It is a a very permissive license, which allows commercial use, modification, distribution, sublicensing, and private use of any code so licensed. It also waives liability for the authors of the code, saving them from needing to worry about lawsuits in cases where their code would otherwise be liable - the code is granted as is, and what the user does with it is not the author's fault. The only restriction is that you need to include the license in any software which uses it.
\item The Apache License 2.0, developed by the Apache Software Foundation\footnote{\href{https://www.apache.org/licenses/}{https://www.apache.org/licenses/}}, is similar, but disallows users from trademarking code with the license, requires a few smaller modifications like stating code changes and adding a NOTICE file, if one exists, to derivational code, and also adds a patents clause for contributors.
\item The BSD licenses were developed for use with Berkeley Software Distribution, a Unix-like OS. There have been multiple iterations; the first, 4-clause license required every subsequent license to reference and acknowledge the original, ending with large lists of acknowledgements; a subsequent 3-clause license (often called the "New" BSD) removed this, but kept a clause which stated that usage does not imply endorsement by the original contributors; and this was removed in a 2-clause version, often called "Simplified" or the "FreeBSD" license. 
\item The GNU General Public License (GPL)\footnote{\href{https://www.gnu.org/licenses/}{https://www.gnu.org/licenses/}} is the main example of copyleft licensing, where any derivative works that use GPL licensed code must also use a GPL license. This causes major issues when users want to combine code from multiple sources, some of whose licenses may conflict. For this reason, the GNU Library or "Lesser" General Public License (LGPL) was created, to allow only code under the LGPL to be accessible and modifiable openly, while all other code doesn't have to be. GPL also demands that users include installation instructions, 
\item Creative Commons licenses\footnote{\href{https://creativecommons.org/licenses/}{https://creativecommons.org/licenses/}}, mostly used for sharing non-code material such as images and documents openly, was created by Lawrence Lessig, the founder of the Creative Commons organisation\footnote{\href{https://creativecommons.org/}{https://creativecommons.org/}}, and may also be used for code projects. There are many licenses they offer, and some variants are copyleft licenses - in particular, "share-alike" clauses are an example of copyleft. 
\item The Unlicense\footnote{\href{https://unlicense.org/}{https://unlicense.org/}}, created in 2010, is another option, which explicitly states that code is unlicensed, with no restrictions, and also with no liability for the authors (unlike code which is not licensed, which has stricter protections under US copyright law than code which specifically excludes a license). There is a Creative Commons Zero\footnote{\href{https://creativecommons.org/publicdomain/zero/1.0/}{https://creativecommons.org/publicdomain/zero/1.0/}} license which is similar, as well as the WTFPL license ("Do What The Fuck You Want Public License")\footnote{\href{http://www.wtfpl.net}{http://www.wtfpl.net}}, which, although intentionally comically profane, is non-trivial in that it is used in 11,714 different software projects on GitHub as of this writing.\footnote{\href{https://github.com/search?q=license\%3AWTFPL}{https://github.com/search?q=license\%3AWTFPL}} 
\end{itemize}

Ultimately, licenses are complicated legal documents with various repercussions for how code is accessible. 

\subsection{Where is open source code?}
\label{subsec:where-is-open-source-code}

For closed source or proprietary software, the code itself often isn't stored in the open or accessible to third parties. However, for open source software to be defined as open source according to OSI's definitions, it needs to be publicly accessible and well-publicised. This means that storing code on a server where it could technically be accessed via some protocol, or less ideally through a mail-order CD as \citet{krauwer2006strengthening} suggested, is not enough; instead, it ought to be linked to elsewhere and available for everyone to access. This raises the question: where is most open source code stored? 

Unequivocally, GitHub\footnote{\href{https://github.com}{https://github.com}} is the largest source of shared, open code on the internet, with 27 million users and 80 million repositories as of March 2018\footnote{\href{https://github.com/about}{https://github.com/about}}. There have been several large-scale studies of its codebase by researchers \citep{gousios2012ghtorrent, allamanis2013mining, gousios2014lean, kalliamvakou2014promises, beller2016analyzing} which confirm this. Other large repositories for code of a similar nature, include Sourceforge, with 430k projects and 3.7m users\footnote{\href{https://sourceforge.net/}{https://sourceforge.net, accessed April 18, 2018}}, Bitbucket\footnote{\href{https://bitbucket.org/}{https://bitbucket.org/}} with 5m users\footnote{\href{https://blog.bitbucket.org/2016/09/07/bitbucket-cloud-5-million-developers-900000-teams/}{https://blog.bitbucket.org/2016/09/07/bitbucket-cloud-5-million-developers-900000-teams/}},  Launchpad\footnote{\href{https://launchpad.net/}{https://launchpad.net/}} with 4.2m users\footnote{\href{https://launchpad.net/people}{https://launchpad.net/people}}, and Gitlab\footnote{\href{http://gitlab.com/}{http://gitlab.com/}}, which holds the majority share of self-hosted Git platforms\footnote{\href{https://about.gitlab.com/is-it-any-good/}{https://about.gitlab.com/is-it-any-good/}}. All of these platforms are based around Git, the versioning software developed by Linus Torvalds, used to store different versions of code for developers and teams, which lends itself particularly to shared code that can be updated easily by outside and community developers. It is worth mentioning that not all of these projects are public. 

Self-hosted Git instances are a common way of storing proprietary code; one sets up a versioning system within a company, using the tools and set of social standards that developers are used to from working on open source code, but limit access to employees. This is what is meant by GitLab's statement that they host most self-hosted Git platforms. Git isn't the only possible versioning software for this; Google has their own versioning tool, Piper, which hosts the over two billion lines of code used by the majority of the company.\footnote{\href{https://www.wired.com/2015/09/google-2-billion-lines-codeand-one-place/}{https://www.wired.com/2015/09/google-2-billion-lines-codeand-one-place/}} Self-hosted Git instances are generally not open source - generally, one is limited to a hosting service such as GitHub, where one pays a fee for hosting, or, more commonly today, hosts for free if public, but pays a fee for private hosting. 

There are alternatives to cloud storage (the cloud here being a common metaphor for hosting on someone else's servers) with a hosting provider; one would be storing the code on one's own website, and running your own server or building the user interface yourself. This is largely uncommon due to setup costs, but occasionally happens with academics and smaller teams who are not used to larger hosts or who are worried about the longevity of providers. This latter worry is founded; for instance, Google Code was closed after ten years of running in 2016, causing many projects to need to port to another service such as GitHub.\footnote{\href{https://code.google.com/}{https://code.google.com/}} For academics, a common solution to offset setup and hosting costs is to use university websites and archives as a suitable place to store open source code. For instance, Giellatekno, a language-technology research group, and Divvun, a linked product development group, both work primarily on S�mi languages, and both use the same Subversion (another versioning system) database for storing their code \citep{moshagenopen}, which is hosted by UiT The Arctic University of Norway.\footnote{\href{http://giellatekno.uit.no/}{http://giellatekno.uit.no/}} 

In a large part, the question of where to store information is one which the large archival sites mentioned in Section~\ref{subsec:finding-resources} was made to solve. In particular, this is true for non-code resources, such as audio and video corpora, which historically have prioritised for storage over code due to the size of the corpora and due to the older industry standards of keeping all code related to research private, especially when that code was funding by enterprise. Many of these sites were repositories of metadata which pointed to individually hosted content, which made the links susceptible to link rot and offloaded the issue of storage altogether. 

Today, however, there is a sea change towards putting computational work in the open. Occasionally, this means that academics point to the open source code for their papers on GitHub or elsewhere, or publish their software itself as a research object. For example, \citet{makela2016integrated} and \citet{kleinberg2017web} were published with the Journal of Open Source Software (JOSS)\footnote{\href{http://joss.theoj.org/}{http://joss.theoj.org/}}, which peer-reviews, publishes, and assigns digital object identifiers (DOIs) to software as a way of recognising important academic work. The code for these papers is publicly available on GitHub. 

\subsection{Digital permanence and storage}
\label{subsec:digital-permanence}

Universities and institutions have short timelines and are largely dependent on specific, allocated, and thus finite funding. Here, I'll answer the question: What other models are there for data storage? What concerns are there?

% Alexis: there are also issues around evolution of code -- what about open source code that does great things but requires, e.g., java from 10 years ago?

\subsection{Data and privacy}
\label{subsec:data-and-privacy}

Here, I'll talk specifically about data rights and privacy, in regards to whether it makes sense to decouple code from data, especially in cases of low resource languages, where sparse data may be naturally enriched with annotation schemas and hard to separate out from the tools being used. In such cases, how do we as a community, researchers as providers, and developers as consumers, deal with licensing, privacy, and proprietary data? Does it make sense to provide links to code that can be used institutionally or commercially without also allowing for things like royalties for usage, or proper licensing for data? Bound up in this are also ethical concerns - well studied in theoretical field linguistics - about the language users themselves not wishing for their data to be used in certain ways.

% From Chiarcos:

% a few years back, I compiled a massive corpus of Bibles and related texts
%in a CES-conformant XML format (following Resnik 1996), some also with
%annotations. For the most part, distributing this corpus would be illegal
%under European copyright law (and that's why you haven't heard about it),
%but I realized that there are circumstances which could allow
%dissemination of a great part of it under an academic license.

% Compiling and distributing a web corpus is basically illegal in Europe
%unless explicitly permitted by an accompanying license. However, US law
%has the concept of fair use, and if a data provider declares US
%legislation to apply (e.g., that "[t]hese Terms and Conditions ... are
%governed by the laws of the State of New York"), we Europeans can rely on
%the principle of fair use, as well.
%
%% According to 17 U.S.C. § 107, "the fair use of a copyrighted work,
%including such use by reproduction in copies or phonorecords or by any
%other means specified by that section, for purposes such as criticism,
%comment, news reporting, teaching (including multiple copies for classroom
%use), scholarship, or research, is not an infringement of copyright." The
%intended use is for NLP research, DH scholarship and classroom use, so
%that would probably not an issue -- and in fact, there is no financial
%damage whatsoever as this data is freely and redundantly available from
%the web.
%
%% However, am I allowed to distribute this corpus with an explicit license
%statement? I think CC-BY-NC should protect the intellectual and commercial
%interests of the creator of the electronic edition and be roughly in the
%spirit of an academic license, but of course, I'm not the actual owner of
%the data, but only responsible for its transformation and annotation. I am
%wondering about the consequences if someone eventually creates an NLP tool
%chain from this data and uses any models trained on the data in a
%commercial application. As the original copyright extends to derived
%works, this would be a clear violation of my license statement, of course,
%but I would be responsible as I redistributed the data and by transforming
%it from messy HTML to proper markup, I actually enabled this violation.


\subsection{Legal rights and liability}
\label{subsec:legal-rights}

Here, I'll talk about specific licenses used in Open Source, and how they apply to code. I'll try to keep this brief.

I'll also talk about liability wavers - a separate issue from licenses. I'll talk about the standard liability wavers used with the MIT license, and other issues that might arise for language code specifically.

\subsection{Military and enterprise solutions}
\label{subsec:military-and-enterprise}

In this section, I will talk about how open source meshes with military and enterprise development.

%% https://www.nist.gov/itl/iad/mig/lorehlt-evaluations, DARPA

\subsection{Funding}
\label{subsec:oss-funding}

Here, I'll talk about funding again - but in terms of open source code. This will be a short section.

% IARPA and DARPA both are involved with low resource languages and both of them may have their own institutional values that are probably at ends with independent researchers, commercial consumers, and language communities. Does working on sparse data openly bring along with it ethical or moral concerns; if so, how can these be adequately explained, breached, and talked about? How can they be worked around or be part of the conversation? Note that DARPA and the like also use humanitarian reasons as their primary stated aim for work on sparse languages, which may be contrary to their military needs. There is already an extensive literature on moral uses of data -- I could summarize that, and apply it specifically to low resource languages, which is something I do not think has yet been published.

% Darpa: http://www.darpa.mil/program/low-resource-languages-for-emergent-incidents
% IARPA: http://www.iarpa.gov/index.php/research-programs/babel

% - Institutional bottleneck
% - Linguistic colonialism
% - Ethical and moral concerns for military usage
% - Ethical and moral concerns for big business usage

\subsection{Ethical reasons for using open source}
\label{subsec:oss-ethics}

Finally, I want to close with a discussion of the moral and ethical reasons for using open source, and whether or not these concerns are relevant to computational linguists.
