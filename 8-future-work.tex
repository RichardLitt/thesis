\section{Future Work}\label{sec:future-work}

Here, I'll talk about where to go next.

\subsection{Beyond Wikipedia and Ethnologue}

I'll talk about the shortcomings of both Wikipedia as a service, and Ethnologue as a provider of language data. Specifically, I want to draw attention to how Wikipedia treats its long-term contributors, and how Ethnologue charges exorbitant fees for using its data, and what we can do to improve this.

% \subsection{An Open Data Repository}

% I'll spec out the plans for an open data repository that could be used to share data.

% - Peer-to-peer solution for sharing code
%   - Stub out example
%   - Build a web searcher for automatically getting and sharing code
%     Further Work:
%   - Open source data repositories (touch on)
%   - Working with Ethnologue


% \subsection{Storage on a p2p network}
%
% Build a web-application tool for serving a decentralized data store for endangered language tools and data
%
% Example:
%
% I have already put a subset of repositories listed on endangered-languages into IPFS, a p2p resource for storing and disseminating data in a decentralized and persistent fashion.
%
% Process:
%
% 1. `cat` the endangered-languages README.md, then `grep` for `/.*(//github\.com/.*?/[a-zA-Z0-9-]*).*/` (all github.com repos).
% 2. Output list into separate file.
% 3. `awk` the first few repos, until a random divider, and clone the git repos: `awk '1;/kuromoji-server/{exit}' ../githublist.md | xargs -n1 git clone`
% 4. `ipfs add -r repos`
% 5. `ipfs pin add repos`
