\section{Methods}\label{sec:methods}

\subsection{Choosing a license}

I'll give some recommendations on a license, both for individuals and for larger companies. I am not a lawyer, so this will be short and tempered.

\subsection{Choosing repositories}

I'll talk about my actual recommendations for storing code. I'll talk about how GitHub is a business, and its aims may not be aligned with researchers interested in long term archival, and similar concerns.

% Longer term plans for open source repositories; GitHub is useful currently, but it also a business, and as such its aims may not be aligned with its users. I would like to talk about building a database of open source repositories on a secure, permanent, peer-to-peer network. This is something I am actively involved in professionally (I currently work at IPFS, which is building such a network). I would like to talk about linguistic and scientific applications of using versioned, p2p, and distributed systems for storing both open source code related to low resource languages as well as language data.

\subsection{Sharing code without a platform}

I'll outline a plan for peer-to-peer resource sharing, using IPFS \citep{benet2014ipfs} and other related tech. I'll mention a case study involving local indigenous communities in Guyana using peer-to-peer to track illegally logging on their land, and explain how this system could also be used for language development.\footnote{\href{https://www.digital-democracy.org/}{https://www.digital-democracy.org/}}
