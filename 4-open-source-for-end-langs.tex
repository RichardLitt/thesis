\section{Open Source Code for Endangered Languages}\label{sec:endlangcode}
% Open Source code for endangered languages
\subsection{BLARK}
\subsection{NLTK and other larger libraries}

\href{NLTK (Natural Language Toolkit)}{http://nltk.org/} is an free and open source library which uses the Python language, and enables users to interface with over fifty different corpora and lexical resources. A primer written by the main creators, \href{Natural Language Processing with Python}(http://nltk.org/book), is used frequently in natural language processing classes written by the creators. It is licensed under the Apache 2.0 license, a common license \footnote{https://github.com/nltk/nltk/blob/develop/LICENSE.txt}. On GitHub, there are currently 204 contributors listed \href{https://github.com/nltk/nltk/graphs/contributors}, although the git history shows 234 (found by using the command `git authors` % TODO Explain
). Some of the resources within NLTK have to do with low resource languages. For instance, in 2015, NLTK added machine translation libraries, including popular ones such as IBM Models 1-3 and BLEU.

By open sourcing their code, the NLTK authors have allowed it to be adapted and re-used. Currently, there are several ports.
% TODO cite.
One of these is the JavaScript language implementation, \href{https://github.com/NaturalNode/natural}{https://github.com/NaturalNode/natural}. This has 6700 stars on GitHub, which is a good indicator of community vitality and use, and 88 contributors. The port is also open source, under an MIT license \href{https://github.com/NaturalNode/natural\#license}{https://github.com/NaturalNode/natural\#license}.

\subsection{Other resources}

\subsection{Why isn't more code open?}

- Longevity of linguistic scholarship and work

\section{A Database for Open Source Code}\label{sec:solutions}

I propose a study of RichardLitt/endangered-languages:
- It's uses (specifically)
- Current considerations in it's planning
- reception
  - User evaluations from other open source scientists
- Future goals


- Case study using endangered-languages repository
  - Clean up resource
    - Add all listed resources in issues
    - Contact and create the LSA CELP Technology Subcommittee
    - Clone all SourceForge repositories
    - Rename to low-resource-languages
    - "List quality"
    - "the pages and subpages are often dead"
  - Get diagnostics on the state of the links I've found:
    - What percentage have been updated when
    - Downloaded, etc.
  - Review Excel results

% Other resources
% http://www.elda.org/en/catalogues/language-resources-announcements/
% http://www.elsnet.org/
