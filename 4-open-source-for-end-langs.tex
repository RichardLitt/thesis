% !TEX root = thesis.tex
\section{Open Source Code for Low Resource Languages}\label{sec:endlangcode}
% Open Source code for endangered languages

In this section, I'll move on to the real meat of this thesis; how is open source code used for computational linguistics, and specifically for LRLs.

\subsection{BLARK and beyond}

First, I am going to talk about BLARK - the Basic LAnguage Resource Kit proposed by \citet{krauwer2003basic} - and what a language needs digitally as a base layer to digitally ascend. I haven't talked specifically about how computational linguistics addresses low resource languages yet - the preceding sections have largely been showing the state of the field and what open source is. We'll get to open source eventually, but here I want to cover the tools needed for a language.

I'll then mention tools here that can be used after a language has some digital presence - basically, what makes an LRL a resourced language.

%% http://www.blark.org/
%% Blark:  The BLARK concept was defined in a joint initiative between ELSNET (European Network of Excellence in Language and Speech) and ELRA (European Language Resources Association) [Krauwer, 1998] and first launched with a Dutch initiative called Dutch Human Language Technologies Platform that was initiated in April 1999. Then, in the framework of the ENABLER thematic network (European National Activities for Basic Language Resources -Action Line: IST-2000-3.5.1), ELDA elaborated a report defining a (minimal) set of LRs to be made available for as many languages as possible and mapping the actual gaps that should be filled in so as to meet the needs of the HLT field.

%% http://www.elsnet.org/dox/blark.html

%% Quote some BLARK papers from various languages - relatively easy to find papers.


\subsection{NLTK and other open source libraries}

Here, I'll explain some open source resources that can be used to bootstrap development; for instance, \href{NLTK (Natural Language Toolkit)}{http://nltk.org/}, a free and open source library which uses the Python language by \citet{bird2006nltk}, and enables users to interface with over fifty different corpora and lexical resources.

% A primer written by the main creators, \href{Natural Language Processing with Python}(http://nltk.org/book), is used frequently in natural language processing classes written by the creators. It is licensed under the Apache 2.0 license, a common license \footnote{https://github.com/nltk/nltk/blob/develop/LICENSE.txt}. On GitHub, there are currently 204 contributors listed \href{https://github.com/nltk/nltk/graphs/contributors}, although the git history shows 234 (found by using the command `git authors` % TODO Explain
% ). Some of the resources within NLTK have to do with low resource languages. For instance, in 2015, NLTK added machine translation libraries, including popular ones such as IBM Models 1-3 and BLEU.

% By open sourcing their code, the NLTK authors have allowed it to be adapted and re-used. Currently, there are several ports.
% % TODO cite.
% One of these is the JavaScript language implementation, \href{https://github.com/NaturalNode/natural}{https://github.com/NaturalNode/natural}. This has 6700 stars on GitHub, which is a good indicator of community vitality and use, and 88 contributors. The port is also open source, under an MIT license \href{https://github.com/NaturalNode/natural\#license}{https://github.com/NaturalNode/natural\#license}.

% \subsection{Other resources}

% % Other stuff
% Not all research that is code based can be easily quantified as open source. For instance, [Afranaph](http://www.africananaphora.rutgers.edu/home-mainmenu-1) is a database of research on African languages. However, there is no code directly available to build your own database. Instead, you only have the option of searching their database. Other sites may use open source technology, but not be open source themselves. For instance, [TransNewGuinea](http://transnewguinea.org/about) has a colophon where they mention that they use Unicode, Django, Bootstrap, jQuery, Leaflet, PostgreSQL, and SQLite.
%
% Keyboard layouts are another area where much i18n work has been focused. Link: https://github.com/HughP/MLKA
%
% - Lack of sharing code or storing it usefully, due to factors: funding, academic cycle, inability, scope, lack of knowledge of domain

% - Specific examples of cross-language applicability of an open source coding library (such as NLTK, or more specifically, family-related usage of parsers or MT models), and what that says about the incentives and use cases for open source libraries.

%% Mention LoReHLT https://www.nist.gov/itl/iad/mig/lorehlt-evaluations
%% The tools used here are permissively licenses and publicly available.

\subsection{Data permanence and interoperability}

Here I'll talk about the inextricable nature of linguistics data and code, and how they are linked.

\subsection{Data and privacy}
\label{subsec:data-and-privacy}

Here, I'll talk specifically about data rights and privacy, in regards to whether it makes sense to decouple code from data, especially in cases of low resource languages, where sparse data may be naturally enriched with annotation schemas and hard to separate out from the tools being used. In such cases, how do we as a community, researchers as providers, and developers as consumers, deal with licensing, privacy, and proprietary data? Does it make sense to provide links to code that can be used institutionally or commercially without also allowing for things like royalties for usage, or proper licensing for data? Bound up in this are also ethical concerns - well studied in theoretical field linguistics - about the language users themselves not wishing for their data to be used in certain ways.

% From Chiarcos:

% a few years back, I compiled a massive corpus of Bibles and related texts
%in a CES-conformant XML format (following Resnik 1996), some also with
%annotations. For the most part, distributing this corpus would be illegal
%under European copyright law (and that's why you haven't heard about it),
%but I realized that there are circumstances which could allow
%dissemination of a great part of it under an academic license.

% Compiling and distributing a web corpus is basically illegal in Europe
%unless explicitly permitted by an accompanying license. However, US law
%has the concept of fair use, and if a data provider declares US
%legislation to apply (e.g., that "[t]hese Terms and Conditions ... are
%governed by the laws of the State of New York"), we Europeans can rely on
%the principle of fair use, as well.
%
%% According to 17 U.S.C. § 107, "the fair use of a copyrighted work,
%including such use by reproduction in copies or phonorecords or by any
%other means specified by that section, for purposes such as criticism,
%comment, news reporting, teaching (including multiple copies for classroom
%use), scholarship, or research, is not an infringement of copyright." The
%intended use is for NLP research, DH scholarship and classroom use, so
%that would probably not an issue -- and in fact, there is no financial
%damage whatsoever as this data is freely and redundantly available from
%the web.
%
%% However, am I allowed to distribute this corpus with an explicit license
%statement? I think CC-BY-NC should protect the intellectual and commercial
%interests of the creator of the electronic edition and be roughly in the
%spirit of an academic license, but of course, I'm not the actual owner of
%the data, but only responsible for its transformation and annotation. I am
%wondering about the consequences if someone eventually creates an NLP tool
%chain from this data and uses any models trained on the data in a
%commercial application. As the original copyright extends to derived
%works, this would be a clear violation of my license statement, of course,
%but I would be responsible as I redistributed the data and by transforming
%it from messy HTML to proper markup, I actually enabled this violation.



\subsection{A database for open source code}\label{sec:solutions}

Here, I'll talk about a database of open source code. Specifically, I'll mention my own work building \href{https://github.com/RichardLitt/endangered-languages}{https://github.com/RichardLitt/endangered-languages}, described first in \citet{CCURL}, and what it contains and who has worked on it with me. I'll cover the main tools, what kind of tools were included, and why I built the database on GitHub in this way.

I'll also include diagnostics on how it has been used and how the tools it mentions have been used - what percentage have been downloaded, and so on.

\subsection{Linked data}

Here, I'll briefly talk about related efforts with the Open Linguistics Working Group's \citep{chiarcos2012open} work on open source data reflected on the semantic web.\citep{chiarcos2013building}


% - Its uses (specifically)
% - Current considerations in its planning
% - reception
%   - User evaluations from other open source scientists
% - Future goals

%
% - Case study using endangered-languages repository
%   - Clean up resource
%     - Add all listed resources in issues
%     - Contact and create the LSA CELP Technology Subcommittee
%     - Clone all SourceForge repositories
%     - Rename to low-resource-languages
%     - "List quality"
%     - "the pages and subpages are often dead"
%   - Get diagnostics on the state of the links I've found:
%     - What percentage have been updated when
%     - Downloaded, etc.
%   - Review Excel results

% Other resources
% http://www.elda.org/en/catalogues/language-resources-announcements/
% http://www.elsnet.org/
