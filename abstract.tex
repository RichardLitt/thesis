\thispagestyle{empty}
\begin{abstract}
  \setlength{\parskip}{2ex plus 0.5ex minus 0.2ex}

  % Please put \noindent before each paragraph of the abstract!

  \noindent Of the roughly seven thousand languages currently spoken, less than fifty have a significant digital presence. In order for a language to be used digitally and to survive, it needs to have computational resources: orthographies, dictionaries, grammars, spell checkers, parsers, and more. Instead of depending on closed-source code from large providers, researchers and communities can leverage open source code to bootstrap digital language development. In this thesis, I discuss the state of the field for low-resource languages, what open source code is and how this methodology can help languages. I provide two cases studies, looking in detail at Gaelic and Naskapi, and I describe a database I have developed (with help from others) of open source code serving these languages. Looking to the future, I discuss steps for helping save languages from virtual extinction.

\end{abstract}
