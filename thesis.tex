% !TEX TS-program = arara
% arara: pdflatex
% arara: bibtex
% arara: pdflatex
% arara: pdflatex

\documentclass[12pt,a4paper]{article}
\usepackage{thesiscommands}

% IGT
% Important; include last
\usepackage{gb4e}

%\drafttrue % or
\draftfalse

\author{Richard Littauer}

% to be executed with: lualatex --shell-escape -synctex=1 -interaction=nonstopmode "complete_thesis_new_2".tex
% last changed on 26.8.16 9h30

\begin{document}

\begin{titlepage}
	\begin{center}

		% Upper part of the page. The '~' is needed because \\
		% only works if a paragraph has started.
		\includegraphics[width=0.25\textwidth]{img/eule.png}~\\[1cm]

		\textsc{\LARGE Saarland  University}\\[0.4cm]
		\textsc{\Large Department of Computational Linguistics}\\[1.5cm]

		\textsc{\LARGE Master's Thesis}\vspace{0.5cm}

		% Title
		\HRule \\[0.55cm]

		{ \huge \bfseries Open Source Code

			 and\vspace{0.2cm}

			 Low Resource Languages}\vspace{0.5cm}

		\HRule \\[1.5cm]

		% Author and supervisor
		\begin{minipage}{0.45\textwidth}
			\begin{flushleft} \large
				\emph{Author:}\\
				Richard \textsc{Littauer}\\
				Matriculation: 2539658
			\end{flushleft}
		\end{minipage}
		\begin{minipage}{0.45\textwidth}
			\begin{flushright} \large
				\emph{Supervisors:} \\
				Prof. Dr. Dietrich \textsc{Klakow}\\
				Prof. Dr. Alexis \textsc{Palmer}\\
				% \emph{Advisors:} \\
				% Dr. Harm \textsc{Brouwer}\\
			\end{flushright}
		\end{minipage}

		\vfill

		% Bottom of the page
		{\large \today}

	\end{center}
\end{titlepage}
\newpage

\thispagestyle{empty}
\begin{abstract}
  \setlength{\parskip}{2ex plus 0.5ex minus 0.2ex}

  % Please put \noindent before each paragraph of the abstract!

  \noindent Of the roughly seven thousand languages currently spoken, less than fify have a significant digital presence. In order for a language to be used digitally and to survive, it needs to have computational resources: orthographies, dictionaries, grammars, spell checkers, parsers, and more. Instead of depending on closed-source code from large providers, researchers and communities can leverage open source code to bootstrap digital language development. In this thesis, I discuss the state of the field for low-resource languages, what open source code is and how this methodology can help languages. I provide two cases studies, looking in detail at Gaelic and Naskapi, and I describe a database I have developed (with help from others) of open source code serving these languages. Looking to the future, I discuss steps for helping save languages from virtual extinction.

\end{abstract}


\newpage

%%% Eidesstattliche Erkl�rung
\thispagestyle{empty}
\noindent\subsection*{Eidesstattliche Erkl\"arung}

\noindent Hiermit erkl\"are ich, dass ich die vorliegende Arbeit selbstst\"andig verfasst und keine anderen als die angegebenen Quellen und Hilfsmittel verwendet habe.\\

\noindent\subsection*{Declaration}

\noindent I hereby confirm that the thesis presented here is my own work, with all assistance acknowledged. \\

\vspace{1cm}

\noindent Richard Littauer

\vspace{.5cm}

\noindent Montr\'eal, \thedate


%%% Eidesstattliche Erkl�rung End

\newpage

\thispagestyle{empty}
\noindent\subsection*{Acknowledgements}

This thesis is based loosely on a paper presented at the LREC CCURL Workshop in July 2016 in Slovenia \citep{CCURL}. Hugh Paterson III was a coauthor on that paper. Jonathan Poitz provided formatting files for \LaTeX, used in this paper. Fritz Van Deventer helped by suggesting nicer fonts. Many, many academics have helped advise me along the way towards this work: Bobbye Pernice, Stefan Thater, Ivana Kruijff-Korbayov\'a and Hans Uszkoreit with their administrative assistance; Mike Rosner and Ray Fabri with a previous iteration on Maltese morphological parsing, which encouraged my interest in low resource languages; Christine Schreyer with discussions about constructed languages as low resource languages; Tyler Schnoebelen, Schuyler Erle, Robert Munro, and others from Idibon who advised on one iteration of this thesis; Matthew Bauer, Graham Leary and Francesca Shaw for discussions on Gaelic; Oksana Choulik, Alice Reed, and Caitlin, Matthew and Hazel Windsor, for many conversations in Schefferville and Kawawachikamach; and finally, Alexis Palmer and Dietrich Klakow, who patiently advised me for years.

This work also draws heavily on an open source repository on GitHub, for which Gina Chiodo, Hugh Paterson III, Liling Tan, Ryan Txanson, Robert Forkel, Aidan Pine, Nick Heindl, Kevin Scannell, Sjur Moshagen, Waldir Pimenta, Joshua Olson, Edwin Ko, Arne Neumann, and Pablo Duboue are all contributors (in order of contributions) as of today, as well as the bots ReadmeCritic, greenkeeper[bot] (run by my friend, Gregor Martynus), and orthographic-pedant (run by Travis Hoppe).

Writing this paper involved using LaTeX\footnote{\href{https://www.latex-project.org/}{https://www.latex-project.org/}. \last{May~1}} typeset with TeXShop\footnote{\href{http://pages.uoregon.edu/koch/texshop/}{http://pages.uoregon.edu/koch/texshop/}. \last{May~1}}; Atom\footnote{\href{https://atom.io/}{https://atom.io/}. \last{May~1}}; iTerm\footnote{\href{https://iterm2.com/}{https://iterm2.com/}. \last{May~1}}; Firefox\footnote{\href{https://www.mozilla.org/en-US/firefox/}{https://www.mozilla.org/en-US/firefox/}. \last{May~1}}; Bash\footnote{\href{https://www.gnu.org/software/bash/}{https://www.gnu.org/software/bash/}. \last{May~1}}; and Mac OSX 10.13\footnote{\href{https://www.apple.com/macos/high-sierra/}{https://www.apple.com/macos/high-sierra/}. \last{May~1}}; among a suite of other closed and open source tools. I used TravisCI\footnote{\href{https://travis-ci.org}{https://travis-ci.org}. \last{May~2}} to ensure link validity, and stored all of the code and files for this thesis on GitHub.\footnote{\href{https://github.com/RichardLitt/thesis}{https://github.com/RichardLitt/thesis}. Last accessed \today.}

Where it was more efficient to refer to a website as a footnote, I have done so; some of the resources thus acknowledged may have a publication that I also could have referred to. I have had to date no proofreaders for this paper, so all errors are mine and mine alone. My apologies.

\newpage

%
\tableofcontents
\thispagestyle{empty}
\addtocontents{toc}{\protect\thispagestyle{empty}}

\newpage
\thispagestyle{empty}
\setcounter{page}{0}
\listoffigures
\setcounter{page}{0}
\listoftables
%\listoflistings % For code; not needed here.
\thispagestyle{empty}
\newpage

\setcounter{page}{1}

% This is a joke. Do not turn on.
% \doublespacing

% !TEX root = thesis.tex
\section{Introduction}
\label{sec:intro}

At least half of the world's 7000-odd languages will be extinct this century \citep{krauss92, grenoble2011cambridge}. Just over half of these languages have writing systems.\footnote{\href{https://www.ethnologue.com/enterprise-faq/how-many-languages-world-are-unwritten-0}{https://www.ethnologue.com/enterprise-faq/how-many-languages-world-are-unwritten-0}. \last{May 1} Note that with all Ethnologue links to language data, there is an eventual paywall which inhibits access. Using a private browser session can normally circumvent this paywall adequately, although I am explicitly not recommending such a workaround here.} It is estimated that fewer than 5\% of the world's languages are used online or have significant digital presence \citep{kornai2013digital} - meaning, basically, that they are used on the web or have substantial digital recordings.

The majority of the world's computational technology has been built using English - with English manuals, English interfaces, and by English speakers. The most prevalent language spoken by users of this technology is also English. There are a few languages - around thirty - with the combination of large populations with internet access, official governmental status, and industrial economies which affords them some native computational technology, in particular on the World Wide Web, the largest global network for sharing code and written material.

English is the undisputed heavyweight as far as global written resources are concerned.\footnote{\href{https://w3techs.com/technologies/history_overview/content_language}{https://w3techs.com/technologies/history\_overview/content\_language}. \last{May 1}} Over half of the web's content is written in English. The next largest languages are Russian, German, Spanish, Japanese, and French - with a combined population of well over a billion speakers. Portuguese, Italian, and Chinese have the next largest amount of content - but each of them only covers between 2\% and 3\% of the web's content - followed by Polish, Turkish, Dutch, and Korean with over 1\%. Suffice to say, the graph of global written content is not skewed towards language diversity as a norm. This is not surprising in any event, as around 90\% of the world's languages are spoken by fewer than 10\% of its people \citep{bernard1992preserving}.

In part, these high resource languages depend upon extant corpora to enable further development of human language technology (HLT). It is difficult for languages which are newcomers to the digital world to get started. % This does not follow well
Put simply: a literacy system affords corpora, and corpora can be used by researchers to either build tools for that language or to adapt tools from other languages. These tools might be spell-checkers, parsers, input systems, or later on speech recognition and generation software, semantic analysers, or machine learning and translation systems, among others. But these tools only become useful as soon as there exists corpora for them; otherwise, such work is premature. Further, high resource languages can sometimes bootstrap their efforts by utilising code from related languages. A parser for French might be adapted for Spanish, given a large corpus to train on; whereas adapting code with scarce data is more difficult.

Most code in the world is probably developed in closed environments with consumer endpoints, by the military or large businesses. For instance, the World Wide Web (from here on, the web), the largest shared corpus of written language, started with support from  the Massachusetts Institute of Technology (MIT) and the Defense Advanced Research Projects Agency (DARPA). (This helps to explain why most of the web is written in English.) Another example would be Google Translate, which uses massive bilingual corpora to provide automatic translation services for free online, but whose code is proprietary and owned by Google.

While the enterprise pathway for language resource development works well for large languages where populations of speakers can be leveraged to provide funding, the majority of the world's languages are not able to develop their own computational resources - either grammars, corpora, or code. Instead, they must rely on small groups of researchers, limited funding, and a grab-bag of written resources when they have them. For instance, the most consistent translations cross-linguistically are of the Christian Bible, which may not reflect the target language's culture.

In this thesis, I examine methodology that can be used by linguists, researchers, and language developers to help their languages "digitally ascend" (as \citet{kornai2013digital} puts it) - to bootstrap their corpora creation, write grammars, transform other language's tools and research to their own languages, and to ultimately enable their communities to speak and share their knowledge computationally. This methodology goes under the broad label of \textit{open source} software. Open source software is code which has been developed and made available for free and under a permissive license, without concessions about how it is to be used or who uses it. This allows coders to use code which they personally have not built without allocating funds for it, thus freeing up significant portions of research and development costs for making tools. At present, the majority of the world's code depends on some level on open source software - for instance, Linux and much of the web depend on open source code.

In the field of computational linguistics, however, there is a deficit of resources which are licensed and available as open source. This largely stems from the need to financially recoup expenses for development, on licenses mandated by research groups or military funders, and on a lack of awareness of how open source code works by developers. Another consideration is that an open source label does not ensure that the code is worth using, maintained, relevant, or in scope for a given domain.

% General comment: it is of course personal preference, but I don't much like the use of future tense in describing the content of sections of a document that exists. I'd suggest rewording, for example:
%- "Below, I go into further depth..."
%- "Section 4 defines what open source is and talks about issues..."
%etc.
%
%Same comment applies in later chapters, I am certain.

Below, I go into further depth about the state of low resource languages (LRL) and computational resources in Section~\ref{sec:endlang}, and what different languages need in order to have digital presence in Section~\ref{sec:resources}. In Section~\ref{sec:open-source}, I define what open source is, and talk about issues relevant to open source code for LRLs. I then in Section~\ref{sec:lrl-code} talk about the state of the open source ecosystem for LRLs online, in particular focusing on a database of open source code that I have built with the help of researchers around the world.

I touch on some specific examples of languages which could benefit from open source code in Section~\ref{sec:case-studies}, focusing on Gaelic, a language with tens of thousands of speakers but few online resources, and Naskapi, a language with only a thousand speakers. The Naskapi case study is informed by original research, as I engaged in field research at the town where most Naskapi live and talked to linguists working on literacy efforts for this language. In Section~\ref{sec:methods}, I discuss how open source can help low resource languages, and in Section~\ref{sec:discussion} I expound further at a high level on what open source enables for linguists and language communities. Finally, in Section~\ref{sec:future-work} and Section~\ref{sec:conclusion} I discuss future work, and offer some concluding remarks.

%% TODO Talk to profs about the lack of ethics clearance for these conversations (mostly casual)

% Listing accomplishments of this thesis
This thesis is, to my knowledge, the only paper that looks specifically at what open source resources there are for low resource languages. I provide a quantitative assessment of the state of the field, and suggest a new type of crowd-sourced, curated, and decentralised database for language resource aggregation. I also discuss three in-depth case studies; one of what a specific problem in computational linguistics looks like when viewed from an open source perspective (the example problem involves using geographical information systems with language co\"{o}rdinates), and two case studies of the state of open source resources for entire languages: Gaelic and Naskapi. The Naskapi chapter also serves as a follow-up to \citepos{jancewicz2002applied} paper, looking at how the Naskapi community has changed technologically in the past fifteen years. Finally, I suggest a novel way of storing language resources in an open source fashion using the decentralised web.

\include{endangered-languages}
% The State of endangered languages and computational linguistics
%\subsection{Defining Endangered Languages}
%\subsection{What are computational resources}
% !TEX root = thesis.tex
\section{Resources}
\label{sec:resources}

It makes sense at this point (if not earlier), to discuss what language resources are. There are two main types of resources: corpora and tools which act on corpora. They are inextricably linked, but the approaches towards building, archiving, and using either differ. In this section, I will answer these questions: What resources are needed to take a language from no resources, to a thriving language with a large digital presence? And what resources are there?

For digital vitalisation, \citet{kornai2015new} proposes working on a pyramid approach: first build a corpus with active and engaged speakers, then l10n and i18n support; then word-level tooling such as spell checkers and morphological analysers; phrase and sentence level tooling such as parsers; and finally speech and character recognition and machine translation. This, in general, follows how most language development progresses. However, a finer-grained understanding of the tools would be illuminating.

\subsection{Types of language resources}

While an exposition of all possible natural language processing tools is beyond the scope of this thesis, it is worth going into some depth about some of them.

\subsubsection{Corpora}

All language resources ultimately depend upon corpora; without data, an algorithm does nothing. And yet not all corpora is the same, either. Data which has been cleaned - often using intensive manual effort and specific tooling - is far more efficient than generic buckets of sound clippings or text, although there are uses for both. Annotated corpora is more useful for specific tools, such as syntactic parsers or for morphological analysers. The type of annotation matters; for instance, interlinear glossed text (IGT) is an industry standard for displaying corpora in academic linguistics by displaying the original datum, a morphosyntactic gloss, and a translation. This is particularly useful for developers of morphological analysers and parsers, who are keen to interpret typological features of a language into their system.

Historically, the majority of corpora has been written corpora, due to the difficulty gathering, cleaning, and sorting large amounts of audio or video files. With the rapid escalation of computing power (mirrored and predicted by Moore's law \citep{schaller1997moore}, which observes that circuit board complexity doubles roughly every two years), and with the advent of large social media sites that allow users to upload their own language data (such as YouTube\footnote{\href{https://www.youtube.com/}{https://www.youtube.com/}. \last{May~2}}), audio and visual corpora are becoming more and more prevalent. Both types of corpora are relevant for LRLs; the former is useful for setting up fonts and characters and implementing them in Unicode, for spell checkers, and so on; while the latter can also be used to begin extracting phonemes \citep{kempton2014discovering, muller2017improving}, or for speech-to-text systems \citep{fraga2015active, fraga2015improving}, among other uses \citep{adams2017automatic}. Indeed, the proceedings of the workshop on Spoken Language Technologies for Under-resourced languages (SLTU), now in its sixth incarnation, show that using audio corpora to bootstrap language models is an active topic.\footnote{\href{http://www.mica.edu.vn/sltu2018/}{http://www.mica.edu.vn/sltu2018/}. \last{May~2}} The possibility of automatically extracting linguistic information from audio recordings has enabled field linguists to focus on recording data now, as well, instead of laboriously spending years transcribing a single language \citep{bird2014aikuma}.

There are other types of corpora. For instance, a bilingual or multilingual corpus is increasingly useful for NLP work on LRLs. By comparing aligned or identical translated texts from a source language, one can deduct systemic knowledge of the target language. When this is combined with typological features, one can swiftly build not just machine translators \citep{lewis2010haitian} but also grammars \citep{bender2016linguistic}. Even basic word lists can be useful in some instances; for instance, the Swadesh list of forty-odd words which are shown to be less likely to change over time \citep{swadesh1955towards} has been incredibly useful for showing language relationships and diachronic change. Another type of corpus would be photos of hand-written material or of particular font faces, for use in optical character recognition, where a knowledge of the alphabet or written resources can be used to develop digitised corpora. On a deeper level, wordnets (which track semantic relationships between various words), thesauri, and other semantically-enriched corpora can also be useful for research and language development.

Depending on the level of annotation and tooling done on a corpus, it often makes sense to include the code which cleaned the corpus in some way with the corpus itself. Using the data may require using a certain program. For instance, \citet{kempton2009finding} describe an automatic allophone induction tool using the TIMIT corpus \citep{garofolo1993darpa}, and they go into some depth discussing the tools needed to parse the well-known corpus, such as the SIL Phonology Assistant,\footnote{\href{https://software.sil.org/phonologyassistant/}{https://software.sil.org/phonologyassistant/}. \last{May~2} (Also an open source project on GitHub, at \href{https://github.com/sillsdev/phonology-assistant}{https://github.com/sillsdev/phonology-assistant}. \last{May~2}), although it wasn't open source when originally reviewed in 2008 \citep{dingemanse2008review}} or the SRILM extensible language modelling toolkit \citep{stolcke2002srilm}. However, unless the code is in some way bundled or its development is well-funded, a paper's combined tooling or workflow often isn't available explicitly.

\subsubsection{Code}
\label{sec:resources-code}

Before getting into the specifics of why code is decoupled from data, and what availability means (for that, see Section~\ref{sec:lrl-code}), it is worth exploring what types of computational resources are common for natural language processing (NLP) work. Here are some examples:

\begin{itemize}
  \item \textbf{ Font codecs}. Where a language has a new alphabet or characters which are not included in a standard alphabet, some of the first resources developed for that language are fonts and type-setting for the script. Often, languages may have a rich literary history but no digitisation of their native script; this was the case for Naskapi, and \citet{jancewicz2002applied,jancewicz2012cree} describe the process of setting up Naskapi Syllabics first for typewriters and later for computers over the past thirty years. This is explored further in Section~\ref{sec:naskapi}.
  \item \textbf{Language recognition}. An Cr\'ubad\'an \citep{scannell2007crubadan} uses trigram analysis to determine language identification for texts crawled from the internet; this means breaking down known wordlists into statistical frequency lists, by selecting three-character strings from a training set and seeing how well they match up on a target corpus. There are other types of language recognition software, using word frequencies, bigram analysis (especially for spoken data), and character recognition, for example.
  \item \textbf{Morphological parsers}. These are used to split words into their component pieces (morphemes).
  \item \textbf{Spell-checkers}. At their simplest, these perform string recognition on a dictionary of known forms. This works particularly well for isolating languages like Hawai'ian, where there is a surfeit of morphological differences for individual words. However, for agglutinative or slot-based languages, this requires a good deal more code under the hood, as the system needs to predict validity for morphemes within a word - both derivational (combinations occurring through historical processes, and often no longer productive for grammatical new forms) and inflectional (productive word-building demanded by morphosyntactic processes).
  \item \textbf{Tokenizers}\footnote{Editorial note: While this thesis generally follows British or Canadian spelling rules, for certain terms the American system is used to conform to popular keywords in the scientific literature.} are used to split strings into tokens. Most often, this involves simply splitting words out of a sentence - to a computer, the space character is no different than an alphanumeric one, and so splitting on the character is important for other code to know where a word ends and begins.
  \item \textbf{Lemmatizers} group together inflected forms under one heading, so that a word with many variants can be identified as the having the same lemma, or root.
  \item \textbf{Part-of-speech (POS) taggers} figure out the syntactic function of a particular word within a given context, and are useful for parsers and for word-sense disambiguation. These are most often rule-based, and involve a knowledge of the language's syntactic functions, as well as depending upon morphological parsing for variant forms.
  \item \textbf{Named Entity Recognition (NER)} involves extracting proper names from a text - for instance, any persons, businesses, times, currencies, and so on. This is particularly useful for parsing large corpora to quickly find relevant features; for instance, extracting a politician's name from years of newspaper corpora is a common task for NLP researchers.
  \item \textbf{Syntactic parsers} are used to understand the syntactic function of words within a sentence or phrase, and are particularly useful for machine translation. However, knowing the syntax is also useful for other tools such as sentiment analysis or NER, as it provides a finer-grained understanding of the text and context.
  \item \textbf{Speech-To-Text (STT)} and \textbf{Text-To-Speech (TTS)} are systems which, understandably, convert written corpora into audio, and vice versa. Generally these involve a fair amount of work and a large corpora, although there are systems which are able to produce reasonably useful systems from scarce data (see discussion above). This is useful for a variety of uses, from automatic transcription to robot voice systems to geographical map guidance.
  \item \textbf{Machine Translation (MT)} systems automatically transfer information encoded in one language into another, and generally involve statistical knowledge of the source and target language and complicated grammars which are either encoded directly or built on universal translation systems. The arguably most common MT system today, Google Translate, originally used statistical machine translation with multilingual aligned texts, but is now switching to a neural network system, using more complicated machine learning algorithms \citep{google2016google}.
\end{itemize}

There are more tools which could be named here. The key point to take away is that language development is neither an easy task involving a weekend's work by a team of volunteers, nor a matter of developing a finite set of tools. Instead, it is a gradated process that involves consistent development and fine-tuning, generally involving dozens if not hundreds of language developers working on various parts of the process. One difficulty in this process is finding out what has been done before, to avoid duplicated work. It is to solve this need that resource aggregators exist.

\subsection{Resource aggregators}
\label{subsec:resource-aggregators}

I have already mentioned that An Cr\'ubad\'an \citep{scannell2007crubadan} is a good location to find monolingual texts from the web; however, this is but one of many corpora that might be of use to linguists, language activists, and to NLP practitioners. To find other resources can be an overwhelming task. \citet{unesco11directory} for instance itemises hundreds of such resources. To help solve this issue, there are a non-trivial number of large organisations and databases where it is possible to find resources - dictionaries, academic references, and occasionally software - on low resource languages. To give more of an idea of what these resources are like, here are some major examples:

\begin{itemize}
\item The Unicode Common Local Data Repository (CLDR) "provides key \\ building blocks for software to support the world's languages, with the largest and most extensive standard repository of locale data available."\footnote{\href{http://cldr.unicode.org/}{http://cldr.unicode.org/}. \last{May~2}} There are dozens of scripts available in Unicode.\footnote{\href{https://www.unicode.org/standard/supported.html}{https://www.unicode.org/standard/supported.html}. \last{May~2}}

\item The Endangered Languages Project (ELP), described above and in \citet{lee2016assessing} and online\footnote{\href{http://www.endangeredlanguages.com/}{http://www.endangeredlanguages.com/}. \last{May~2}} has information on many under resourced languages.

\item Ethnologue, which is both a book \citep{lewis2009ethnologue} and an online resource,\footnote{\href{https://www.ethnologue.com/}{https://www.ethnologue.com/}. \last{May~2}} is the most comprehensive resource describing the world's languages, such as population size and the general geographic locations of speakers. It is published by SIL International, an evangelical Christian non-profit organisation, and has proprietary paywalls for repeated access to content. Many SIL entries for specific languages include academic references.

\item Glottolog\footnote{\href{http://glottolog.org/}{http://glottolog.org/}. \last{May~2}} is an open source alternative to Ethnologue, developed at the Max Planck Institute for Evolutionary Anthropology. It has over 180,000 references, with information on over eight thousand languages.\footnote{The astute reader will note that this is more than the amount of languages mentioned in \citet{lewis2009ethnologue}. The definition of what constitutes a language differs, and so the numbers can fluctuate between sources.} \citep{hammarstrom2015glottolog}

\item Omniglot, "the online encyclopaedia of writing systems and languages",\footnote{\href{http://omniglot.com}{http://omniglot.com}. \last{May~2}} contains information on writing systems for around a thousand languages. \citep{ager2008omniglot}

\item The Online Database of Interlinear Text (ODIN)\footnote{\href{http://odin.linguistlist.org}{http://odin.linguistlist.org}. \last{May~2}} is a multilingual repository of annotated language data for 1274 languages.\footnote{Noted as of January 13, 2010 at \href{http://odin.linguistlist.org}{http://odin.linguistlist.org}. \last{May~2}} The database is formed by crawling scholarly articles on the web and looking for IGT examples. As well, "ODIN was developed as part of the greater effort within the GOLD Community of Practice \citep{farrar2007gold} and the Electronic Metastructure for Endangered Languages Data efforts (EMELD),\footnote{\href{http://emeld.org/}{http://emeld.org/} (\last{May~2}) and \citet{farrar2002common}} whose goals are to promote best practice standards and software, specifically those that facilitate interoperation over disparate sets of linguistic data." \citep{lewis2010developing}

\item The Open Language Archives Community (OLAC) is a worldwide virtual library of language resources \citep{simons2003open},\footnote{\href{http://www.language-archives.org/}{http://www.language-archives.org/}. \last{May~2}} and contains thousands of cross-references to resources both on the web and in print form.

\item Wikipedia,\footnote{\href{https://www.wikipedia.org/}{https://www.wikipedia.org/}. \last{May~2}} "the largest and most popular general reference work on the Internet" \citep{wiki:Wikipedia} has a nontrivial amount of articles on low-resource languages, many of which have references themselves to scholarly work. \citet{kornai2013digital}, among others, notes that Wiki\-pedia is one of the first ports-of-call for new language communities, and while it is not a precondition for having corpora on the web, it is a {\it sine qua non} for digital vitalisation. Thus Wikipedia has two purposes; documenting the language and its community (for instance, in the Naskapi Language article\footnote{\href{https://en.wikipedia.org/wiki/Naskapi\_language}{https://en.wikipedia.org/wiki/Naskapi\_language}. \last{May~2}}), and providing a space for corpus development in the target language itself (for instance, as in the Gaelic wikipedia\footnote{\href{https://gd.wikipedia.org/wiki/G\%C3\%A0idhlig}{https://gd.wikipedia.org/wiki/G\`aidhlig}. \last{May~3}}).

\item The World Atlas of Language Structures (WALS) \citep{wals} is a directory of typological features which also includes academic references for many of the over two thousand languages presented. WALS is a curated resource, largely made by a team of 55 experts, and hosted by the Max Planck Institute for Evolutionary Anthropology (the same as Glottlog, and as other resources such as PHOIBLE\footnote{\href{http://phoible.org/}{http://phoible.org/}. \last{May~2}} \citep{phoible} and DOBES\footnote{\href{http://dobes.mpi.nl/}{http://dobes.mpi.nl/}. \last{May~2}} \citep{wittenburg2003dobes} related to taking an inventory of language structures).
\end{itemize}

There are other resources: the CLARIN Virtual Language Observatory,\footnote{\href{https://vlo.clarin.eu}{https://vlo.clarin.eu}. \last{May~2}} the Linguistic Data Consortium at UPenn,\footnote{\href{https://www.ldc.upenn.edu/}{https://www.ldc.upenn.edu/}. \last{May~2}} the ELRA,\footnote{\href{http://catalog.elra.info/en-us/}{http://catalog.elra.info/en-us/}. \last{May~2}} META-SHARE,\footnote{\href{http://www.meta-share.org/}{http://www.meta-share.org/}. \last{May~2}} the Association for Computational Linguistics' Wiki,\footnote{\href{https://aclweb.org/aclwiki/Main_Page}{https://aclweb.org/aclwiki/Main_Page}. \last{May~2}} the NICT Universal Catalogue,\footnote{\href{https://www.nict.go.jp/index.html}{https://www.nict.go.jp/index.html}. \last{May~2}} LT World\footnote{\href{http://www.lt-world.org/}{http://www.lt-world.org/}. \last{May~2}} and so on. Providing an exhausting list would be an impossible task, as there are often collections for each specific language family or region. For instance, Afranaph\footnote{\href{http://www.africananaphora.rutgers.edu/home-mainmenu-1}{http://www.africananaphora.rutgers.edu/home-mainmenu-1}. \last{May~3}} is a database of research on African languages. More pertinently, now that it is clear what resources are, and that it is possible to at least get a basic idea of what resources there are for a given language, what resources are relevant to low resource languages?

\subsection{BLARK and LRE maps}
\label{subsec:blark-and-lre-maps}

\citet{soria2017digital} briefly mention `digital language survival kits' as one of the motivations for their paper - these are explicated more fully on the Digital Language Diversity Project's site.\footnote{\href{http://www.dldp.eu/en/content/digital-language-survival-kit}{http://www.dldp.eu/en/content/digital-language-survival-kit}. \last{May~2}} This project is an EU initiative, through the Erasmus+ programme, and it aims to identify needs and provide `kits' for certain European low resource languages - specifically Basque, Breton, Karelian and Sardinian.

The use of the word `kit' is informative, as there is  pre\"{e}xisting literature on this topic regarding the BLARK, or Basic Language Resource Kit. BLARK was developed by a joint initiative between the European Network of Excellence in Language and Speech (ELSNET), a European international umbrella for 145 different organisations in 29 countries, and the European Language Resources Association (ELRA), and first outlined in \citet{krauwer1998elsnet}. The BLARK is defined as the "minimal set of language reosources that is necessary to do any precompetitive research and education at all." \citep[4]{krauwer2003basic} In general, this comprises "written language corpora, spoken language corpora, mono- and bilingual dictionaries, terminology collections, grammars, modules (e.g. taggers, morphological analysers, parsers, speech recognisers, text-to-speech), annotation standards and tools, corpus exploration and exploitation tools, bilingual corpora, etc."

\citet{krauwer2003basic} has a comprehensive matrix in the appendix outlining technology that would be needed to provide a BLARK for Dutch, as outlined in a workshop documented in \citet{binnenpoorte2002towards}. In another paper, \citet{maegaard2006blark} under NEMLAR (Network  for  Euro-Mediterranean  LAnguage  Resources) outlined the specific language resource needs for Arabic in a BLARK table, noting the importance of certain modules for better language coverage. Both of the BLARK grids for Arabic provided in that paper are included here, in Figures~\ref{fig:blark1} and \ref{fig:blark2}, as they very usefully show not only the state of HLT resources for Arabic at the time, but also the categories thought sufficient. They also point out how both written and spoken language tools need to be developed, and cannot be considered in isolation. Their categories - `prosody prediction', `alignment', `shallow parsing', and so on - are all terms which refer to a suite of resources that each reflect hundreds of papers from within the computational linguistics community, which Section~\ref{sec:resources-code} briefly explicated.

\begin{figure}
 \centering
 \includegraphics[width=1\textwidth]{img/blark1.png}
 \caption{A BLARK graph for Arabic, with written language applications and corresponding HLT modules, marked with importance \citep[775]{maegaard2006blark}}
 \label{fig:blark1}
\end{figure}

\begin{figure}
 \centering
 \includegraphics[width=1\textwidth]{img/blark2.png}
 \caption{A BLARK graph for Arabic, with speech language applications and corresponding HLT modules, marked with importance \citep[776]{maegaard2006blark}}
 \label{fig:blark2}
\end{figure}

The BLARK process - auditing a language, using a grid to identify what corpus and resource needs are necessary for language resources - has now been applied to Swedish \citep{elenius2008language} and Bulgarian \citep{simov2004language}, and numerous South African languages \citep{grover2011south}, among others.

Unfortunately, BLARK (or ELARK, purportedly a more sophisticated version of BLARK for industry described in \citet{mapelli2003report}, according to \citep{grover2011south}) is a large grid, and may not work for languages without extensive funding models or support. For this, there is a smaller BLARK version, the BLARKette, which should work for low resource languages (although how a smaller version of a minimal set could be provided usefully is not clear).

\begin{quote}
In order to accommodate this problem we have proposed the definition of a scaled down, entry-level version of the BLARK, targeting exclusively the research and (especially) the education community. It should be light and compact, not too demanding in terms of hard and software requirements, cheap, free from IPR issues, and ideally small enough to fit on a CD or DVD. We expect to release a first document, with tentative summary specifications, towards the end of 2006. Check the ELSNET site for news. \citep{krauwer2006strengthening}
\end{quote}

The model of transportation for this - a CD, instead of a downloadable resource - shows that the concept has not aged well. There is also a surfeit of references of BLARK or BLARKette in the past decade in the literature - \citet{krauwer1998elsnet} only has 31 references on Google Scholar (an imperfect but effective metric).\footnote{\href{https://scholar.google.com/scholar?cites=5069727220703395724}{https://scholar.google.com/scholar?cites=5069727220703395724}. \last{May~2}} What happened? It is most likely (in my opinion) that building a BLARK for a language is too complex for language groups to perform, and lacks proper incentives. It requires an authoritative and intimate knowledge of a language's space by many researchers, all of whom must come together to identify gaps, often from proprietary institutions. This is a difficult task.

But this effort, in some sense, has expanded into LRE (Language Resources and Evaluation) maps within Europe. As described in \citet{calzolari2010lrec, del2014lremap, mariani2015language, del2015visualising}, the Language Resources and Computation (LREC) conference organisers began asking conference participants who had submitted papers to fill out basic language resource grids when submitting papers. This effort was extended to ten different computational linguistics conferences, covering most large European languages and four regional Spanish languages. This data has been collected into matrices and a database that reflects language resources for a variety of languages. To date, this is the most comprehensive review of NLP per language that I am aware of, with 4395 entries - however, it is worth noting that it is limited in scope. The 133 less-common languages represented in the LRE map represent only 414 entries. An example of the matrix for the high resource languages can be seen in Figure~\ref{fig:lre}, which is a map of resources for various languages, cut off with a lower bound of 50 citations per resource type.

\begin{figure}
 \centering
 \includegraphics[width=1\textwidth]{img/lre.png}
 \caption{LRE maps for high resource languages \citep[460]{mariani2015language}}
 \label{fig:lre}
\end{figure}

Several authors working on LRE maps are also authors of the \citet{soria2017digital} paper; extending the LRE maps for low resource languages, and then intensifying efforts to develop low-hanging fruit for low resource languages is a logical next step for this research. The focus on European languages is expected; this may stem from the fact that LREC, the main conference series from which LRE data was drawn, is run by the European Language Research Association (ELRA). This fragmentation of the field is unsurprising, and happens in the reverse, as well: for example, \citet{paricio2010new} cites a framework for upgrading low resource languages which is explained in a research paper written in Spanish, and, anecdotally, around half of the papers presented at the Ryukyuan Heritage Language Society's conference in Tokyo in 2012 (which I attended) were presented in Japanese. This is not to say that fragmentation and diversity of linguistics in academia is something to be avoided, but rather that it is a hurdle to be noted and worked with to avoid repeated work and splintered efforts.

% Note: Made redundant by other examples in the text. I do not think this is needed.
%\subsection{The current state of language diversity}
%In this section, I am going to briefly go into detail about what diversity means for linguistics. This will be useful later for explaining how related languages can be used to bootstrap work in similar languages. For instance, Irish spell-checkers and constitutional corpora from the EU can be used by Scottish Gaelic speakers with some tweaks in order to further improve their own systems.

\subsection{Who makes resources for languages?}
\label{subsec:who-makes-resources}

Different groups  work on different stages of language development, and each brings their own perspective, intentions, tools, and achievements. Abstractly, the groups could most easily be separated into language communities and linguists, and the fields of computational linguistics and NLP. The first group are those - often not computational linguists by training or NLP researchers - who want their own language or the language they are studying to exist digitally in some form. The initial step is generally to adopt any language script, whether  pre\"{e}xisting or ready-made for the language by linguists (for examples of this, see the Endangered Alphabets Project\footnote{\href{http://www.endangeredalphabets.com/}{http://www.endangeredalphabets.com/}. \last{May~2}}) into Unicode, a standard for consistent character representation.\footnote{\href{https://unicode.org/}{https://unicode.org/}. \last{May~2}} There are linguistic research groups that focus on this problem; for instance, the Script Encoding Initiative at Berkeley.\footnote{\href{http://linguistics.berkeley.edu/sei/index.html}{http://linguistics.berkeley.edu/sei/index.html}. \last{May~2}}

Some of the people involved in this process may be computational linguists. \citet{bender2016linguistic} makes a distinction between the fields of computational linguistics and NLP: "computational linguistics is used to describe research interested in answering linguistic questions using computational methodology, while natural language processing describes research on automatic processing of human language for practical applications." It should be clear here that computational linguistics is a subfield of linguistics, and that the two are not always in sync, as for instance \citet{kay1997proper} points out when discussing improving MT processes by using informed linguists to build semi-supervised systems. \citet{bender2010grand, bender2016linguistic} go further, suggesting that understanding language typology can drastically help with multilingual NLP.

Meanwhile, many experts in NLP would not consider themselves computational linguists, but developers, just as many language developers would not consider themselves linguists. While navigating the field or looking at resources, it is important to keep these distinctions in mind, as they inform narratives concerning resource generation, scope, and efforts.

Another hurdle which was briefly alluded to earlier was the plethora of large organisations, databases, or projects dedicated to cataloguing low resource languages. Each of these has differences in scope, funding, and incentives. However, large organisations are not the only groups working on language development, digital ascent, language revitalisation, or any other shared focus that relates to low resource languages.

As \citet{hammarstrom2015unesco} points out, "language documentation and description is an extremely decentralized activity, carried out by missionaries, anthropologists, travellers, naturalists, amateurs, colonial officials, ethnographers and not least linguists over several hundred years." Language communities, amateur and professional linguists, educators, and language policy setters are most often involved in standardising a language and helping to document and revitalise low resource languages. Digitally, amateur computational linguists and coders who are first-language speakers of their respective LRLs are often the first to work on translating or migrating resources; this group is also often the first to set up wikipedias in a local language (although this often leads to enthusiastic loners working outside of the main language communities \citep{soria2017digital}). These researchers do not necessarily work only on their own languages; for instance, the {\it Indigenous Languages and Technology} email listserv\footnote{\href{http://www.u.arizona.edu/~cashcash/ILAT.html}{http://www.u.arizona.edu/~cashcash/ILAT.html}. \last{May~3}} connects hundreds of different people interested in this area of research, without focusing on a single language.

Beyond these groups, universities and local governments can also often develop language resources for low resource languages, as was the case with \citet{rognvaldsson2009icelandic} and with First Voices, a Canadian First Nations archiving platform.\footnote{\href{https://fv.nuxeocloud.com/}{https://fv.nuxeocloud.com/}. \last{May~3}} Non-profits are also common in the language resource development space; examples include the Endangered Language Alliance,\footnote{\href{http://elalliance.org/}{http://elalliance.org/}. \last{May~3}} Terralingua,\footnote{\href{http://terralingua.org/}{http://terralingua.org/}. \last{May~3}} the Foundation for Endangered Languages,\footnote{\href{http://www.ogmios.org/index.php}{http://www.ogmios.org/index.php}. \last{May~3}} the Living Tongues Institute for Endangered Languages,\footnote{\href{https://livingtongues.org/}{https://livingtongues.org/}. \last{May~3}} and so on.

After these groups, large grant-driven institutions such as CLARIN or the NSF fund a large portion of language development, along with industry giants such as Google or Xerox, and large military research arms such as DARPA. Sometimes grants come from philanthropic organisations, like the Bill and Melinda Gates foundation \citep{penfield2006technology}, or philanthropic arms of enterprises, such as Rosetta Stone's work on some indigenous North American languages.\footnote{\href{https://www.rosettastone.com/endangered/projects}{https://www.rosettastone.com/endangered/projects}. \last{May~3}}

Unfortunately, the lion's share of the overall funding for language development goes to languages which are already resourced.

\begin{quote}
Over the years the EU has invested massively in the development of language and speech technology, and many dedicated R\&D programmes have had a significant impact on its advancement, including applications oriented towards solving the multilinguality problem... Unfortunately the strong industrial bias of recent EU programmes has led to a situation where the major part of the funding for language and speech technology goes to the major languages. This is not surprising, as industrial players will prefer to invest in the development and deployment of technologies for larger markets. As a consequence there has been only marginal support for the development of language and speech technology for the language communities that do not constitute profitable markets. As the development cost of such technologies is independent of the number of speakers of a language ("all languages are equally difficult") this has created a very unbalanced situation. \citep{krauwer2006strengthening}
\end{quote}

Or:

\begin{quote}
Were it not for the special attention DARPA, one of the main sponsors of machine translation, devoted to Haitian Creole, it is dubious we would have any MT aimed at this language. There is no reason whatsoever to suppose the Haitian government would have, or even could have, sponsored a similar effort \citep{spice}. \citep[9]{kornai2013digital}
\end{quote}

The attentions of DARPA and IARPA (Intelligence Advanced Research Proj\-ects Activity, a US institution modelled after DARPA but focusing on national security and not military concerns) on low resource languages is clear, and these are only two branches of the US military-industrial complex. The Low Resource Languages for Emergent Incidents (LORELEI),\footnote{\href{https://www.darpa.mil/program/low-resource-languages-for-emergent-incidents}{https://www.darpa.mil/program/low-resource-languages-for-emergent-incidents}. \last{May~3}} Machine Translation for English Retrieval of Information in Any Language (MATERIAL)\footnote{\href{https://www.iarpa.gov/index.php/research-programs/material}{https://www.iarpa.gov/index.php/research-programs/material}. \last{May~3}} and Babel\footnote{\href{https://www.iarpa.gov/index.php/research-programs/babel}{https://www.iarpa.gov/index.php/research-programs/babel}. \last{May~3}} projects explicitly mention `low resource languages` as research areas. The budget for these projects is not public, but can be safely assumed to be in the tens of millions of dollars.\footnote{\href{https://www.popularmechanics.com/technology/security/a17451/iarpa-americas-secret-spy-lab/}{https://www.popularmechanics.com/technology/security/a17451/iarpa-americas-secret-spy-lab/}. \last{May~3}} As well, the general public cannot know how these tools and projects are then utilised, and whether it is in the best interest of speakers of low resource languages. It is safe to assume that, unless the speakers are US citizens or allies, they are not, as the American military's stated goal is to ensure security for its own nationals. This means that any work done by these bodies, and by researchers connected to them, can be seen as controversial.

Another good example of where funding and incentives for language development can be controversial would be Ethnologue, which rate limits and has a paywall guarding usage of their database, even though they are widely recognised as one of the best informed databases for language data. This paywall can be triggered by viewing a non-trivial amount of pages (around five) for languages on the site. While this is philosophically no different than asking researchers to buy \citet{lewis2009ethnologue}, this data is the most widely used, and this paywall was only implemented in late 2015.\footnote{\href{https://www.ethnologue.com/ethnoblog/m-paul-lewis/ethnologue-launches-subscription-service\#.VmA0hspth0o}{https://www.ethnologue.com/ethnoblog/m-paul-lewis/ethnologue-launches-subscription-service\#.VmA0hspth0o}. \last{May~3}}

SIL International is also the standardising body in charge of the ISO 639-3 standard, which is the most widely used language code. By having a paywall on their data, they exclude the general public from having control of codes for their own languages. SIL has also come under criticism for their Christian missionary work, as it can be viewed as complicit in culture change, and by extrapolation, ethnocide \citep{dobrin2009sil, dobrin2009practical, everett2009don}. This is just one example - and most likely one of the most extreme, not counting military work on languages used by insurgents in wars (implied above) - of how organisations working on language resources may influence the work itself.

The funding of language resource development matters, because the way that the language community approaches language development affects the chance of survival for the language. This is one of the reasons that \citet{grenoble2016response} pointed out that `language vitality' is a more politically correct term to use than `language endangerment', as it takes the focus away from loss and focuses attention on language ascent. Another reason that language funding matters is because the major players with funding will generally be able to out manoeuvre smaller groups with different resources. This can enforce language shift, and can render resources created by individual developers moot. For instance, the secwepemc-facebook\footnote{\href{https://github.com/kscanne/secwepemc-facebook}{https://github.com/kscanne/secwepemc-facebook}. \last{May~2}} tool developed to automatically translate Facebook into low resource languages, created by the late Neskie Manuel for his native Secwepemcts\'in, is no longer an active project and has not been updated, rendering it obsolete with Facebook UI changes, while automatic translation is provided for high resource languages natively by Facebook. Scannell, who helped port the secwepemc-facebook tool to Greasemonkey\footnote{\href{https://www.greasespot.net/}{https://www.greasespot.net/}. \last{May~3}} for use on Mozilla Firefox\footnote{\href{https://www.mozilla.org/en-US/firefox/}{https://www.mozilla.org/en-US/firefox/}. \last{May~3}} was one of the authors of \citet{streiter2006implementing}, which suggested that developers for low resource languages use open source software pools in order to pool resources to enable them to overcome this - among other - issues facing low resource languages in particular.

As in Section~\ref{subsubsec:response}, covering all of the potential issues with funding and the politics of language development is well beyond the scope of this paper. However, focusing on how open source can help low resource languages is not. But first; what do I mean by `open source'?

\include{floss}
% \subsection{Defining "open source"}
% \subsection{Where is open source code?}
% \subsection{Data rights and privacy}
% \subsection{Liability}
% \subsection{Funding}
% \subsection{Military and enterprise solutions}
% \subsection{Ethical reasons for using open source}
% !TEX root = thesis.tex
\section{Open Source Code for Low Resource Languages}
\label{sec:endlangcode}

Now that low resource languages have been described, and now that there has been a brief overview of open source as a software methodology, the reader will doubtless wonder - what is the state of open source code that can be used today by language communities?

% Removed as we've covered this in the Resources chapter, enough
% \subsection{BLARK and beyond}
% \label{subsec:blark-and-beyond}

Unfortunately, due to the decentralised nature of open source, this is an inherently difficult question to answer. In the ecosystem, there are a few strategies that can be used to inform an answer: use a specific task as a case study for what tools would be used, look at what resources are available from any of the main large data aggregators mentioned in Section~\ref{subsec:resource-aggregators}, take a screenshot of the ecosystem based on some of the more-cited open source tool used for low resource language NLP, examine linked open data, and sample relevant work on GitHub through a manually collected list of resources. Each of these strategies is employed in a subsection, below.

\subsection{Case study: Mapping linguistic co\"ordinates}

The breadth of HLT is wide; choosing a specific task within it and then trying to perform that task as adequate as possible would be one way to figure out how much open source code exists, and what that looks like. For example, suppose we were interested in making dialect maps using language co\"ordinates. This is an old research problem in linguistics, and computational methods for mapping languages have been described in some research. % TODO cite "The Journal of Linguistic Geography8 was launched in 2013, and provides recognition that mapping is not simply a factual representation, but a core element of the ?...pursuit of a better understanding of the nature of language structure and language change? (Labov & Preston 2013:1)." 

For NLP, this is a nontrivial task, as l10n within a browser can depend upon geographical information from the user. For instance, if the client's browser does not send a {\tt Accept-Language} header\footnote{\href{https://tools.ietf.org/html/rfc7231\#section-5.3.5}{https://tools.ietf.org/html/rfc7231\#section-5.3.5}} in their requests to view a website, specifying languages the client understands by using ISO 639 tags\footnote{\href{https://www.ietf.org/rfc/bcp/bcp47.txt}{https://www.ietf.org/rfc/bcp/bcp47.txt}}, then the server may use the {\tt NavigatorLanguage.language} object in JavaScript\footnote{\href{https://www.w3.org/TR/html51/webappapis.html\#language-preferences}{https://www.w3.org/TR/html51/webappapis.html\#language-preferences}} to query for the language of the browser UI (normally set by the users depending on where they downloaded it), or they could ask the browser directly through the geolocation API (for instance, on Firefox\footnote{\href{https://developer.mozilla.org/en-US/docs/Web/API/Geolocation/Using\_geolocation}{https://developer.mozilla.org/en-US/docs/Web/API/Geolocation/Using\_geolocation}}) to supply the geolocation of users and extrapolate plausible languages from this data. Knowing where the user is likely to be, and what languages the user is likely to prefer using, could help with providing their native language automatically in the browser.

\citet{gawne2016mapmaking} gives a general overview of the mapping field currently, pointing out that the main resource for finding language geographical co\"ordinates comes from the World Language Mapping System\footnote{\href{http://www.worldgeodatasets.com/language/.}{http://www.worldgeodatasets.com/language/. Last accessed April~25, 2018.}}, a website owned and run by SIL, which are used for ISO 639-3 labelling, and by Glottolog and OLAC. The maps are under a closed license and must be purchased. \citet{gawne2016mapmaking} also mention WALS, which uses its own geographical co\"ordinates, and the ELP, which understandably uses Google Maps as its mapping program, and draws from multiple sources. They also mention Language Landscape,\footnote{\href{http://www.languagelandscape.org/}{http://www.languagelandscape.org. Last accessed April~25, 2018.}} a project which maps instances of language use on a map.

To use these geographic information systems (GIS), one needs to download licensed map data, which could be open or closed. Then, one has to have a mapping software to display that data. This software must also be appropriately licensed. Google Maps is not open source, although it is {\it open access}, in that it is free to use. An open source equivalent of Google Maps is Open Street Maps,\footnote{\href{https://www.openstreetmap.org/}{https://www.openstreetmap.org/}} a community built tool that is permissively licensed as CC-BY-SA.\footnote{\href{https://www.openstreetmap.org/copyright}{https://www.openstreetmap.org/copyright}} One could use data from Glottolog or the ELP and then map provide a map using Open Street Map while using entirely open source applications, but the end result could be reproduced on Google Maps with the same lack of restrictions - the only difference is that the engine making Google Maps would be a black box. 

It is this mixed use case that is most common - researchers or NLP practitioners use a mix of open and closed resources, as needed. \citet{gawne2016mapmaking} mention many programs: Google Earth\footnote{\href{https://www.google.com/earth/}{https://www.google.com/earth/}} (closed source, free) for base maps; Geotag\footnote{\href{http://geotag.sourceforge.net/}{http://geotag.sourceforge.net/}} (free, open source) and Photo KML\footnote{\href{http://www.visualtravelguide.com/Photo-kml.html}{http://www.visualtravelguide.com/Photo-kml.html}. This URL was provided in \citet{gawne2016mapmaking}, but may be down permanently.} (free) for accessing GIS embedded in pictures taken on iPhones (closed); the KML and KMZ formats,\footnote{\href{http://www.opengeospatial.org/standards/kml/}{http://www.opengeospatial.org/standards/kml/}} originally developed by Google for Google Earth but now standards implemented by the Open Geospatial Consortium\footnote{\href{http://www.opengeospatial.org}{http://www.opengeospatial.org}} and licensed openly and freely; Koredoko\footnote{\href{https://itunes.apple.com/us/app/koredoko-exif-and-gps-viewer/id286765236}{https://itunes.apple.com/us/app/koredoko-exif-and-gps-viewer/id286765236}} for viewing GIS data in photos (closed, free); CartoDB\footnote{\href{http://cartodb.com}{http://cartodb.com}} (proprietary) and CartoCSS\footnote{\href{https://github.com/mapbox/carto}{https://github.com/mapbox/carto}} (free, open); TileMill\footnote{\href{http://www.mapbox.com/tilemill}{http://www.mapbox.com/tilemill}} (free, open, but no longer maintained or updated) and MapBox\footnote{\href{https://www.mapbox.com/mapbox-studio/}{https://www.mapbox.com/mapbox-studio}} (open, freemium) ; QGIS\footnote{\href{http://www.qgis.org}{http://www.qgis.org}} (free, open); the SQL\footnote{\href{https://www.iso.org/committee/45342/x/catalogue/p/1/u/0/w/0/d/0}{https://www.iso.org/committee/45342/x/catalogue/p/1/u/0/w/0/d/0}} language (free, open - languages and formats also have licensing laws and can be copyrighted\footnote{Interestingly, constructed natural languages can also be licensed and copyrighted, leading to legal complications involving corporations suing fan communities for publishing documentation in a given language. Further discussion is out of scope here.}); JPEG\footnote{\href{https://www.iso.org/standard/54989.html}{https://www.iso.org/standard/54989.html}} and PNG\footnote{\href{https://www.iso.org/standard/29581.html}{https://www.iso.org/standard/29581.html}} image formats (free, open); Adobe PhotoShop\footnote{\href{https://www.adobe.com/products/photoshop.html}{https://www.adobe.com/products/photoshop.html}} (closed source, paid); and CartoHexa\footnote{\href{http://www.colorhexa.com}{http://www.colorhexa.com}} (free, closed).

An example of a mixed workflow would be using a closed source application or website to shim open source data. For example:

\begin{quote}
To give some more general locational context we downloaded some Open Access geopolitical boundaries for Nepal from the Global Administrative Areas website.\footnote{\href{http://gadm.org/download.}{http://gadm.org/download.}} This data was downloaded as KMZ, which TileMill cannot read, so we opened the files in Google Earth (remember ... that KMZ is a compressed KML) and resaved them as KML, which TileMill can read. \citep[228]{gawne2016mapmaking}
\end{quote}

This particular use-case may have benefited from a specific tool which could convert KMZ to KML. A cursory look on GitHub shows 54 repositories that could be relevant\footnote{\href{https://github.com/search?p=1&q=kmz+kml&type=Repositories}{https://github.com/search?p=1\&q=kmz+kml\&type=Repositories}}, including one which does solely this task (albeit with Spanish documentation).\footnote{\href{https://github.com/fadamiao/kmz2kml}{https://github.com/fadamiao/kmz2kml}} Using an entirely open source pipeline for working with language (or GIS data, as here) is rare, although it is hypothetically possible; however, one quickly runs into problems where open source is concerned, as each subsequent layer of computational processing must then depend upon open source - including the operating system (for instance, GNU/Linux as an open source alternative to the closed Mac OS), processor, silicon chips, and so on. (This is one of the reasons that copyleft remains an issue in licensing.) Idiomatically put: there are turtles all the way down. 

As \citet{hu2012multimedia, hu2018web} notes, the general trend in mapping software has been away from native (meaning on the OS level) applications and towards web applications, which may have a steeper learning curve, but which afford remote storage and access, and users over the Internet. WALS uses LeafletJS\footnote{\href{http://leafletjs.com/}{http://leafletjs.com/}}, an open source mapping software that uses Open Street Maps as an alternative to using an embedded Google Maps map using their API. \citet{hu2018web} suggests a workflow that uses Leaflet along with jQuery\footnote{\href{https://jquery.com/}{https://jquery.com/}}, an open source JavaScript utility library, to display GIS linguistic maps. Further study around using only FLOSS software for displaying GIS data for linguistics is necessary.

\citet{cenerini2017mapping} cite several open source software applications and libraries they used in their study mapping the Cree-Innu-Naskapi continu\"{u}m using data from the Algonquian Linguistic Atlas \citep{junker2011linguistic},\footnote{\href{http://www.atlas-ling.ca/}{http://www.atlas-ling.ca/}} but don't open source their own code. This would have been useful, specifically as replicating their study using R \citep{ihaka1996r} would require researchers to write all of their own queries again. More on data privacy will be discussed in Section~\ref{subsec:data-and-privacy}.

This was a small example, looking at only a couple of papers and showing how following open source methodology can be difficult, and how using mixed source applications is often necessary for research and linguistic information. This was a single use case, and every application involving NLP requires navigating software and licensing laws. My purpose in providing this study was to point out how describing the state of open source code that could be used for LRLs is not clear cut. 

There are cases where it is decidedly clear cut. For instance, if the goal is to build a part of speech tagger using two hours of annotation, you could use the low-resource-post-tagging-2014 package developed as part of \citet{garrette2013real,garrette2013learning}, and available on GitHub\footnote{\href{https://github.com/dhgarrette/low-resource-pos-tagging-2014}{https://github.com/dhgarrette/low-resource-pos-tagging-2014}} without any other considerations than downloading Java and learning a bit of Scala, both free and open source languages. But this is a very limited use case, as this package was built as part of two scientific papers studying this narrowly scoped area.

\subsection{LRL NLP available through data providers}
\label{subsec:lrl-nlp-through-providers}

The first resource aggregator listed in Section~\ref{subsec:resource-aggregators} starts on the lower end of the language resource pyramid: Unicode's CLDR resources. Unicode is often the first port-of-call for a language team working on developing scripts for their language, unless the script is already using some  pre\"{e}xisting format (such as the Roman alphabet). CLDR has instructions on checking out their open source subversion repository online.\footnote{\href{http://cldr.unicode.org/index/downloads}{http://cldr.unicode.org/index/downloads. Last accessed on April~24, 2018.}} They also have a GitHub repository\footnote{\href{https://github.com/unicode-cldr/cldr-json}{https://github.com/unicode-cldr/cldr-json. Last accessed on April~24, 2018.}} and organisation with code for digesting the normally XML representation in JSON, the notation format used most often by JavaScript developers. However, CLDR isn't an aggregator - it's more of a suite of tools under one umbrella, as the scope is limited to working with the Unicode format.

Finding resources isn't easy. The Endangered Languages Project, for instance, contains information on over 3000 languages, and has 6830 languages.\footnote{\href{http://www.endangeredlanguages.com/resources/}{http://www.endangeredlanguages.com/resources. Last accessed on April~24, 2018.}} None of these resources are code: the searchable formats are: Format, Image, Video, Document, Audio, Link, Guide. Glottolog only has academic references, and ODIN only has interlinear glossed text (IGT) corpora. Omniglot describes alphabets but doesn't index tooling for them. CLARIN has thousands of resources - but none of them are code, and you need to be an accredited researcher from a European institution to access them. The LDC has a tool page, where it notes five tools that may be useful for researchers using its data. The ELRA site provides hundreds of resources - mostly corpora - for purchase.

% TODO: Pick up at FLOSS LRL LR aggregators tabs

% http://www.elda.org/en/catalogues/language-resources-announcements/
% http://www.elsnet.org/

\subsection{Popular open source libraries}
\label{subsec:popular-open-source-libraries}

\subsubsection{The Natural Language Toolkit}
Here, I'll explain some open source resources that can be used to bootstrap development; for instance, \href{NLTK (Natural Language Toolkit)}{http://nltk.org/}, a free and open source library which uses the Python language by \citet{bird2006nltk}, and enables users to interface with over fifty different corpora and lexical resources.

A primer written by the main creators, \href{Natural Language Processing with Python}(http://nltk.org/book), is used frequently in natural language processing classes written by the creators. It is licensed under the Apache 2.0 license, a common license \footnote{https://github.com/nltk/nltk/blob/develop/LICENSE.txt}. On GitHub, there are currently 204 contributors listed \href{https://github.com/nltk/nltk/graphs/contributors}, although the git history shows 234 (found by using the command {\tt git authors}).% TODO Explain
Some of the resources within NLTK have to do with low resource languages. For instance, in 2015, NLTK added machine translation libraries, including popular ones such as IBM Models 1-3 and BLEU.

By open sourcing their code, the NLTK authors have allowed it to be adapted and re-used. Currently, there are several ports. % TODO cite.
One of these is the JavaScript language implementation, \href{https://github.com/NaturalNode/natural}{https://github.com/NaturalNode/natural}. This has 6700 stars on GitHub, which is a good indicator of community vitality and use, and 88 contributors. The port is also open source, under an MIT license \href{https://github.com/NaturalNode/natural\#license}{https://github.com/NaturalNode/natural\#license}.

\subsection{Other resources}

% % Other stuff
Not all research that is code based can be easily quantified as open source. For instance, Afranaph\footnote{\href{http://www.africananaphora.rutgers.edu/home-mainmenu-1}{http://www.africananaphora.rutgers.edu/home-mainmenu-1}} is a database of research on African languages. However, there is no code directly available to build your own database. Instead, you only have the option of searching their database. Other sites may use open source technology, but not be open source themselves. For instance, TransNewGuinea\footnote{\href{http://transnewguinea.org/about}{http://transnewguinea.org/about}} has a colophon where they mention that they use Unicode, Django, Bootstrap, jQuery, Leaflet, PostgreSQL, and SQLite.

Keyboard layouts are another area where much i18n work has been focused. Link: https://github.com/HughP/MLKA

Lack of sharing code or storing it usefully, due to factors: funding, academic cycle, inability, scope, lack of knowledge of domain

Specific examples of cross-language applicability of an open source coding library (such as NLTK, or more specifically, family-related usage of parsers or MT models), and what that says about the incentives and use cases for open source libraries.

\subsubsection{i18n documentation for larger open source tools}

In some cases, tools themselves may canonically be used for NLP, but may also be translated into LRLs, thus allowing low resource language developers to use the code themselves for bootstrapping their tools. For example, Node has an i18n and l10n committee that works to translate tools - and there is some interested in working with LRLs. % Blah blah blah maybe this is a stretch?
Another instance would be code which has been ported into rare languages % cite uspanteko work in java

% Removed as we cover this, basically, in Open Source
%\subsection{Data permanence and interoperability}
%\label{subsec:data-permanence-and-interoperability}
%
%Here I'll talk about the inextricable nature of linguistics data and code, and how they are linked.
%
%% TODO Cite Austin Principles for Data AustinPrinciples2017

\subsection{Data and privacy}
\label{subsec:data-and-privacy}

Here, I'll talk specifically about data rights and privacy, in regards to whether it makes sense to decouple code from data, especially in cases of low resource languages, where sparse data may be naturally enriched with annotation schemas and hard to separate out from the tools being used. In such cases, how do we as a community, researchers as providers, and developers as consumers, deal with licensing, privacy, and proprietary data? Does it make sense to provide links to code that can be used institutionally or commercially without also allowing for things like royalties for usage, or proper licensing for data? Bound up in this are also ethical concerns - well studied in theoretical field linguistics - about the language users themselves not wishing for their data to be used in certain ways.

% TODO
% Look at ethics section http://www.unesco.org/new/fileadmin/MULTIMEDIA/HQ/CLT/pdf/International%20cooperation%20programs.pdf

% Ethical issues o2010ethical
% Mention decolonialization cushman2013wampum

% From Chiarcos:

% a few years back, I compiled a massive corpus of Bibles and related texts
%in a CES-conformant XML format (following Resnik 1996), some also with
%annotations. For the most part, distributing this corpus would be illegal
%under European copyright law (and that's why you haven't heard about it),
%but I realized that there are circumstances which could allow
%dissemination of a great part of it under an academic license.

% Compiling and distributing a web corpus is basically illegal in Europe
%unless explicitly permitted by an accompanying license. However, US law
%has the concept of fair use, and if a data provider declares US
%legislation to apply (e.g., that "[t]hese Terms and Conditions ... are
%governed by the laws of the State of New York"), we Europeans can rely on
%the principle of fair use, as well.
%
%% According to 17 U.S.C. § 107, "the fair use of a copyrighted work,
%including such use by reproduction in copies or phonorecords or by any
%other means specified by that section, for purposes such as criticism,
%comment, news reporting, teaching (including multiple copies for classroom
%use), scholarship, or research, is not an infringement of copyright." The
%intended use is for NLP research, DH scholarship and classroom use, so
%that would probably not an issue -- and in fact, there is no financial
%damage whatsoever as this data is freely and redundantly available from
%the web.
%
%% However, am I allowed to distribute this corpus with an explicit license
%statement? I think CC-BY-NC should protect the intellectual and commercial
%interests of the creator of the electronic edition and be roughly in the
%spirit of an academic license, but of course, I'm not the actual owner of
%the data, but only responsible for its transformation and annotation. I am
%wondering about the consequences if someone eventually creates an NLP tool
%chain from this data and uses any models trained on the data in a
%commercial application. As the original copyright extends to derived
%works, this would be a clear violation of my license statement, of course,
%but I would be responsible as I redistributed the data and by transforming
%it from messy HTML to proper markup, I actually enabled this violation.

Sometimes, privacy revolves less around the users or the language communities, and more around researchers not wishing to open source their code until they are done developing their project, or until a grant ends, or until they are safe that they won't be scooped by other researchers. For instance, in a paper describing a tool for sharing interlinearized and lexical data in different formats, \citet{kaufman2018kratylos} notes that "Kratylos will be made open-source and accessible to the public through a GitHub repository at the end of the current grant period. Kratylos is built entirely from open- source software itself and transcodes proprietary media formats into the open-source codecs Ogg Vorbis (for audio) and Ogg Theora (for video)." This is particularly insightful, as it shows that an understanding of open source can still be tied to initial closed-source development. % TODO Is there a name for this model, of closed then open?

%% TODO Extend section on academics staying private

\subsection{Linked open data}
\label{subsec:lod}

Here, I'll briefly talk about related efforts with the Open Linguistics Working Group's \citep{chiarcos2012open} work on open source data reflected on the semantic web.\citep{chiarcos2013building}

\subsection{A database for open source code}
\label{sec:solutions}

Here, I'll talk about a database of open source code. Specifically, I'll mention my own work building \href{https://github.com/RichardLitt/endangered-languages}{https://github.com/RichardLitt/endangered-languages}, described first in \citet{CCURL}, and what it contains and who has worked on it with me. I'll cover the main tools, what kind of tools were included, and why I built the database on GitHub in this way.

I'll also include diagnostics on how it has been used and how the tools it mentions have been used - what percentage have been downloaded, and so on.


% - Its uses (specifically)
% - Current considerations in its planning
% - reception
%   - User evaluations from other open source scientists
% - Future goals

%
% - Case study using endangered-languages repository
%   - Clean up resource
%     - Add all listed resources in issues
%     - Contact and create the LSA CELP Technology Subcommittee
%     - Clone all SourceForge repositories
%     - Rename to low-resource-languages
%     - "List quality"
%     - "the pages and subpages are often dead"
%   - Get diagnostics on the state of the links I've found:
%     - What percentage have been updated when
%     - Downloaded, etc.
%   - Review Excel results


% TODO Mention caballero.pdf
% This was a linguistic fieldwork exercise; they made their own tools for resource management in ELAN https://github.com/ucsd-field-lab/kwaras In the paper, they use Word, Quicktime, ELAN, GitHub... but there's not much open source code, at all. Most of it is shimmied together.


% Open Source code for endangered languages
% \subsection{BLARK}
% \subsection{NLTK and other larger libraries}
% \subsection{Other resources}
% \subsection{solutions}
% A database for open source code
% Describe github.com/RichardLitt/endangered-languages
% !TEX root = thesis.tex
\section{Case Studies}
\label{sec:case-studies}

After having done a broad review of open source code for low resource languages above, here I dive deeper by looking for resources for two languages in particular: Scottish Gaelic and Naskapi. Both of these are living languages with speaking communities, although their size, coverage by academic research, and political situations are slightly different. Searching for resources for a specific language is most likely the most common use-case for users interested in LRLs, especially as the majority of LRL researchers work with a single language or a suite of languages that they use themselves, as opposed to researchers working on quantitative studies of languages in general. A deep dive should illuminate how open source methodologies can drive language development.

\subsection{Scottish Gaelic}
\label{sec:gaelic}

Scottish Gaelic (G\`aidhlig is the autonym) is a Celtic language spoken by roughly 60,000 people mainly in the United Kingdom and to a lesser extent in Canada. Gaelic - sometimes called Scots Gaelic, simply Gaelic, or the Gaelic - is a Goidelic or Q-Celtic language, along with Manx and Irish (also sometimes called Irish Gaelic, but here always referred to as Irish). This means that, while related to the Brythonic languages of Welsh, Cornish and Breton, it is different enough to not be able to benefit from the many resources available in Welsh, which, while endangered, has a much stronger academic interest and presence in the United Kingdom, with roughly half a million speakers. Gaelic has traditionally been heavily repressed, both politically and culturally, which has lead to its usage in largely restricted or rural areas, and in the domains of the house, church, and family \citep{mackinnon1991past}.

The 2011 Scottish Census indicates that out of the total amount of Gaelic speakers, only around half - 32,191 person to be exact - read and write in Gaelic.\footnote{\href{http://www.scotlandscensus.gov.uk/}{http://www.scotlandscensus.gov.uk/}. \last{April~27}} 6,218 speak and read the language, but do not write it, while 4,646 can read it, but do not speak or write it. Gaelic officially is not a national language, although it is afforded certain protections under the European Charter for Regional or Minority Languages\footnote{\href{https://www.coe.int/en/web/conventions/full-list/-/conventions/treaty/148}{https://www.coe.int/en/web/conventions/full-list/-/conventions/treaty/148}. \last{April~27}} (although, as this is an EU charter, it is unclear whether Britain will continue to ratify it following their impending exit from the European Union). The Gaelic Language (Scotland) Act of 2005 (GLS) gave Gaelic official status as an official language of Scotland,\footnote{\href{http://www.legislation.gov.uk/asp/2005/7}{http://www.legislation.gov.uk/asp/2005/7}. \last{April~27}} and set up the B\`ord na G\`aidhlig\footnote{\href{http://www.gaidhlig.scot/}{http://www.gaidhlig.scot/}. \last{April~27}} as a language developmental body tasked with protecting and vitalising the Gaelic language.

The B\`ord officially is tasked with promoting and facilitating educational materials, but the initial charter makes no mention of language technology. The National Gaelic Language Plan 2018-2023\footnote{\href{http://www.gaidhlig.scot/launch-of-the-new-national-gaelic-language-plan/}{http://www.gaidhlig.scot/launch-of-the-new-national-gaelic-language-plan/}. \last{April~27}} \citep{bord2018national} mention the Digital Archive of Scottish Gaelic (DASG),\footnote{\href{http://dasg.ac.uk/en}{http://dasg.ac.uk/en}. \last{April~27}} the largest corpus project for Gaelic, but do not specify other language technology being developed (excepting a brief mention of working with Ireland and Nova Scotia developing shared technology and resources). There are some primary and secondary schools, as well as various Gaelic Language and Studies degrees at English-speaking universities, as well as one Gaelic-speaking university Sabhal M\`or Ostaig\footnote{\href{http://www.smo.uhi.ac.uk/en/}{http://www.smo.uhi.ac.uk/en/}. \last{April~27}} on Skye; educational material from the B\`ord is mainly focused in these areas.

\subsubsection{Language vitality status}
\label{sec:gaelic-vitality-status}

Gaelic has an EGIDS rating of 2, as it is a provincial language given the 2005 GLS Act.\footnote{\href{https://www.ethnologue.com/language/gla}{https://www.ethnologue.com/language/gla}. \last{April~27}}  \citet{lewis2009ethnologue} note regarding language use that "Resurgence of interest in Scottish Gaelic in 1990s. A number of children learn the language but there are serious problems in language maintenance even in the core areas \citep{salminen2007endangered}. Home, church, community." UNESCO judges it to be {\it definitely endangered}.\footnote{\href{http://www.unesco.org/languages-atlas/en/atlasmap/language-iso-gla.html}{http://www.unesco.org/languages-atlas/en/atlasmap/language-iso-gla.html}. \last{April~27}} The Endangered Languages Project describes it as Threatened or Vulnerable, depending on the source,\footnote{\href{http://endangeredlanguages.com/lang/3049}{http://endangeredlanguages.com/lang/3049}. \last{April~27}} as \citet{salminen2007europe} gives a much smaller population number of 20k for speakers than the other census-based data. \citepos{kornai2013digital} rating declares it as {\it Living}.\footnote{\href{https://hlt.bme.hu/en/dld/language/4656}{https://hlt.bme.hu/en/dld/language/4656}. \last{April~27}} These ratings are summarized in Table~\ref{table:gaelic}.

\begin{table}
\centering
\begin{tabular}{|p{5cm}|p{5cm}|} \hline
{\bf Scale} & {\bf Grade} \\ \hline
UNESCO & Definitely endangered\\ \hline
Ethnologue (EGIDS) & 2 (Provincial) \\ \hline
LEI & Threatened or Vulnerable \\ \hline
Kornai & Living \\ \hline
\end{tabular}
\caption{Scale for Gaelic}
\label{table:gaelic}
\end{table}

\subsubsection{Language resources}
\label{subsec:gaelic-resources}

Gaelic has a long, written history. Today, there are a plethora of written, audio, and video resources. Some of these have been bundled into linguistic corpora. The DASG is the largest corpus for Gaelic available on the web; however, it is not permissively licensed for modification, distribution, or reproduction, and so cannot be considered open source (although it is open access).\footnote{\href{http://dasg.ac.uk/about/terms/en}{http://dasg.ac.uk/about/terms/en}. \last{April~27}} OLAC has 26 resources for Gaelic, including large multilingual corpora, as well.\footnote{\href{http://www.language-archives.org/language/gla}{http://www.language-archives.org/language/gla}. \last{April~27}} A large corpus compiled by An Crub\'ad\'an is available online \footnote{\href{http://crubadan.org/languages/gd}{http://crubadan.org/languages/gd}. \last{April~27}} \citep{scannell2007crubadan}. WALS has 61 typological features listed for Gaelic,\footnote{\href{http://wals.info/languoid/lect/wals_code_gae}{http://wals.info/languoid/lect/wals\_code\_gae}. \last{April~27}} and Glottolog 35 references.\footnote{\href{http://glottolog.org/resource/languoid/id/scot1245}{http://glottolog.org/resource/languoid/id/scot1245}. \last{April~27}} ODIN has 59 IGT entries for Scottish Gaelic.\footnote{\href{http://odin.linguistlist.org/}{http://odin.linguistlist.org/}. \last{April~27}}

Some of these corpora are annotated - for instance, the Annotated Reference Corpus of Scottish Gaelic (ARCOSG)\footnote{\href{https://datashare.is.ed.ac.uk/handle/10283/2011}{https://datashare.is.ed.ac.uk/handle/10283/2011}. \last{April~27}} \citep{ARCOSG2016, lamb2014scottish}, which used an Irish POS tagger \citep{ui2006part} to project annotations, and which was funded by the B\`ord na G\`aighlig. This resource was used to automatically derive categorial grammars \citep{batchelor2016automatic}, and to develop POS taggers directly for Gaelic \citep{lamb2014developing}. A dependency-structure corpus is being developed \citep{batchelor2014gdbank}, as are word-embedding models \citep{lamb2016developing}. The source code for \citet{batchelor2014gdbank, batchelor2016automatic} is available on GitHub.\footnote{\href{https://github.com/colinbatchelor/gdbank/}{https://github.com/colinbatchelor/gdbank/}. \last{April~27}} Some of these papers were presented at the first Celtic Language Technology Workshop in Dublin in 2014. The amount of resources show clearly that Gaelic is not entirely on the fringe of academic research, although it is generally considered a low resource language.

\citet{scannell2007crubadan} and contributors\footnote{\href{http://crubadan.org/acknowldegments}{http://crubadan.org/acknowldegments}. \last{May~2}} used the Cr\'ubad\'an corpus to create an open source Hunspell spellchecker,\footnote{\href{https://github.com/kscanne/hunspell-gd}{https://github.com/kscanne/hunspell-gd}. \last{April~27}} which is the spellchecker for "LibreOffice, OpenOffice.org, Mozilla Firefox 3 and Thunderbird, Google Chrome, and it is also used by proprietary software packages, like macOS, InDesign, memoQ, Opera and SDL Trados."\footnote{\href{https://hunspell.github.io/}{https://hunspell.github.io/}. \last{April~27}} This spellchecker was built with the help of Michael Bauer, an independent Gaelic technologist who runs a small Gaelic technology consultancy called Am Faclair Beag,\footnote{\href{http://www.faclair.com/}{http://www.faclair.com/}. \last{April~27}} and also has ports for OpenOffice directly\footnote{\href{https://addons.mozilla.org/ga-IE/firefox/addon/scottish-gaelic-spell-checker/}{https://addons.mozilla.org/ga-IE/firefox/addon/scottish-gaelic-spell-checker/}. \last{April~27}} and a Firefox extension.\footnote{\href{https://extensions.openoffice.org/en/project/faclair-afb}{https://extensions.openoffice.org/en/project/faclair-afb}. \last{April~27}} An Faclair Beag also offers an online dictionary with over 85k words\footnote{\href{http://www.faclair.com/GaelicDictionaryAbout.html\#About}{http://www.faclair.com/GaelicDictionaryAbout.html\#About}. \last{April~27}}
 (and almost a million forms\footnote{\href{http://www.faclair.com/News.html}{http://www.faclair.com/News.html}. \last{April~27}}) and in inbuilt lemmatizer.\footnote{\href{http://www.faclair.com/News.html}{http://www.faclair.com/News.html}. \last{April~27}} Another spellchecker also exists on GitHub,\footnote{\href{https://github.com/gooselinux/hunspell-gd}{https://github.com/gooselinux/hunspell-gd}. \last{April~27}} but it is probably derivative, and it has not been worked on recently.

 More complicated, higher level technology also exists. Previous academic work on Gaelic text-to-speech systems (TTS) stretches back at least 20 years; a diphone text-to-speech system for Gaelic was developed, for instance, in 1997, by \citet{wolters1997diphone}, although that is not open source. Today, there is a proprietary synthetic TTS system called Ceitidh\footnote{\href{https://www.cereproc.com/en/CereProc_Gaelic_Synthetic_Voice_Ceitidh}{https://www.cereproc.com/en/CereProc\_Gaelic\_Synthetic\_Voice\_Ceitidh}. \last{April~27}} (pronounced `Katie'), created by a private Gaelic company together with funding from the Scottish Government and the B\`ord na G\`aidghlig. Although Ceitidh is available to developers and students at a reduced or free fee, it is not entirely open source. There are almost no open source sound resources. The main reason is that there is no overall quality assurance for Gaelic sound uploaded online. For large languages, this is not a problem; however, for smaller languages, the size of the corpus means that much of the content may come from only a few sources, none of which may be ideal. This issue may involve general lack of relevance of sound files, or poor quality recordings, or any dialect or non-mainstream features slipping in. Ceitidh was based on original audio files from Kirsteen MacDonald (in Gaelic, Kirsteen NicDh\`{o}mhnaill), some of whose content (while not vetted by an independent linguist) are available on LearnGaelic.scot,\footnote{\href{https://learngaelic.scot/}{https://learngaelic.scot/}. \last{April~27}} which could be hypothetically used to build an open source TTS system. However, quality assurance would be an arduous step.

Navigating resources to identify what is open source and what is not is difficult. As mentioned in Section~\ref{subsec:where-is-open-source-code}, one of the OSI's definitions for open source was that it be well publicised. This cannot be said to be the case for coding resources for Gaelic; there is no central location for viewing tools. The LRE Map has no Gaelic resources, although a POS Tagger, two corpora, a tokenizer, and Babouk corpus tool resource are mentioned for Irish.\footnote{\href{http://www.resourcebook.eu/searchll.php}{http://www.resourcebook.eu/searchll.php}. \last{April~27}} Linghub returns 30 entries - not many, considering it is an aggregator.\footnote{\href{http://linghub.org/search/?query=Gaelic}{http://linghub.org/search/?query=Gaelic}. \last{April~27}} GitHub returns 62 repositories that mention Gaelic,\footnote{\href{https://github.com/search?q=gaelic}{https://github.com/search?q=gaelic}. \last{April~27}} although it is unclear if these are for Irish.

The best resource is arguably Kornai's lab page\footnote{\href{https://hlt.bme.hu/en/dld/language/4656}{https://hlt.bme.hu/en/dld/language/4656}. \last{April~27}} (again, in development). While not linking directly, it does give some information. It notes that there are: several language packs at the OS level for Ubuntu and Windows input, but not one for Mac, probably because Gaelic uses the Roman alphabet and a UK keyboard suffices for most needs;\footnote{I use the US International Keyboard with OSX to type Gaelic accents, myself, and have never needed another keyboard layout for this} a large Wikipedia; a Hunspell checker; OLAC texts (with marginally out of date numbers); a large Cr\'ubad\'an corpus (1,541,302 words and 17,308 documents), as well as a large Indigenous Tweets corpus with half a million words; and general coverage in Omniglot,\footnote{\href{http://omniglot.com/writing/gaelic.htm}{http://omniglot.com/writing/gaelic.htm}. \last{April~27}} bible.org,\footnote{\href{https://bible.org/}{https://bible.org/}. \last{April~27}} Panlex,\footnote{\href{https://panlex.org/}{https://panlex.org/}. \last{April~27}} and the Leipzig corpora \citep{goldhahn2012building}.\footnote{\href{http://wortschatz.uni-leipzig.de/en/download/}{http://wortschatz.uni-leipzig.de/en/download/}. \last{April~27}} Some of the stats are dubious. For instance, 15k wikipedia users seems odd for a language where there a total population of 30k literate speakers; and it is in WALS, quite clearly. However, in general, this gives a better overview than any other source.

As far as I am aware, the highest amount of code resources for Gaelic which are directly linked and open source is the corpus described in Section~\ref{sec:solutions}. There are six resources mentioned in the list,\footnote{\href{https://github.com/RichardLitt/low-resource-languages\#scottish-gaelic}{https://github.com/RichardLitt/low-resource-languages\#scottish-gaelic}. \last{April~27}} which was largely sourced by manually inspecting each of the GitHub repositories mentioning "Gaelic", and also through personal curation during general research for this paper.

Ideally, researchers would start to open source more of their code involving Gaelic. However, there are so few researchers and language communities currently working on Gaelic HLT that this may be a na\"{i}ve wish. Indeed, the main researcher over the past decade for Gaelic releases most of his code publicly - that is, Kevin Scannell of \citet{scannell2007crubadan}. And his focus is mainly on Irish. One solution would be to implement a Scottish Gaelic computational linguistics course at one of the major Scottish universities, such as the University of Edinburgh, Glasgow, St. Andrews, or potentially at Sabhal M\`or Ostaig. This option would reward further lines of inquiry.

% TODO Mention Scannell's pooling paper
% TODO Mention Salt on a shoestring

\subsection{Naskapi}
\label{sec:naskapi}

In October 2017 I travelled to Kawawachikamach and informally interviewed linguists working on a Naskapi Bible, visited the school and talked to teachers at length about language efforts there, and talked to individual Naskapi speakers about their thoughts on the language and how it is used. Below, I give a brief overview of Naskapi, note how it would be rated according the metrics covered in Section~\ref{subsec:metrics}, and discuss language resource development. \citet{jancewicz2002applied} is the main source of published information on Naskapi computational developments; I give an update, 15 years on, given my experience in Kawawachikamach. %I was unfortunately unable to meet Bill Jancewicz, the SIL missionary there, at that time.

\subsubsection{Language background}
\label{sec:naskapi-language-background}

Naskapi (autonymically \sylla{naskapi} naskapi or \sylla{iyuw iyimuun} iyuw iyimuun) is a Cree language in the Algonquin family spoken in central Quebec \citep{MacKenzie-and-Jancewicz-1994}. Virtually the entire population of around 900 Naskapi live within the reservation Kawawachikamach, around 10 miles from Schefferville, QC. There is another Naskapi community on the Labrador coast, who speak another dialect known as Mushuau Innu, which is out of scope of this paper. Schefferville is only accessible by train or plane, and contains another local tribe called the Innu (which has more than 17,000 members, scattered among Quebec and Labrador\footnote{\href{https://en.wikipedia.org/wiki/Innu}{https://en.wikipedia.org/wiki/Innu}. \last{April~27}}), who live on their own reservation and who speak Montagnais or Innu-aimun, a related language. The two languages are similar, and the Naskapi youth are often diglossic in Montagnais (but the Innu are often not) \citep{macKenzie1980towards}.

The Naskapi speak English as a first or second language, while the Innu speak French (and some speak three or all four languages). They moved to Kawawachikamach in the 1960s, after initially being resettled in Schefferville in the early 1950s. Some of the elders still remember being a nomadic people who followed caribou and were raised in the bush. However, half of the population is under the age of 16, and nationally the First Nations population is the largest growing population in Canada.\footnote{\href{http://www12.statcan.gc.ca/census-recensement/2016/dp-pd/index-eng.cfm}{http://www12.statcan.gc.ca/census-recensement/2016/dp-pd/index-eng.cfm}. \last{May~2}}

All of the Naskapi speak their own language regularly, in all contexts - excepting, perhaps, digitally. In the schools, there are Naskapi-only classes held until Grade 8 \citep{llewellyn2017oral}. While there are a few social workers, teachers, and nurses who speak solely English, most jobs in Kawawachikamach are held by Naskapi. There has been a long tradition of missionaries, and almost all of the Naskapi are Protestant. At church, they use Montagnais hymnals and a Montagnais bible.

\subsubsection{Language vitality status}
\label{sec:naskapi-vitality-status}

\citet{lewis2009ethnologue} classifies Naskapi as Level 4 (educational), and notes that "Literacy rate in L1: Western Naskapi: 50\%. Literacy rate in L2: 50\%. Ongoing community language program in Western Naskapi. All children through [sic] in kindergarten through grade 6 can read and write in the language (2017 N. Jancewicz).\footnote{This was gathered through personal communication with Norma Jean Jancewicz, one of the SIL missionary married couples together with Bill Jancewicz (SIL personal communication, 2018).} Taught in primary schools in Western Naskapi. Dictionary. Grammar. NT: 2007. "\footnote{\href{https://www.ethnologue.com/language/nsk}{https://www.ethnologue.com/language/nsk}. \last{April~27}} UNESCO defines it as {\it vulnerable}.\footnote{\href{http://www.unesco.org/languages-atlas/en/atlasmap/language-id-2354.html}{http://www.unesco.org/languages-atlas/en/atlasmap/language-id-2354.html}. \last{April~27}} \citepos{kornai2013digital} digital vitality index awkwardly declares it to be {\it dead}.\footnote{\href{https://hlt.bme.hu/en/dld/language/5651}{https://hlt.bme.hu/en/dld/language/5651}. \last{April~27}} Naskapi does not appear at all on the Endangered Languages Project. These ratings are displayed in Table~\ref{table:naskapi}.

\begin{table}
\centering
\begin{tabular}{|p{5cm}|p{5cm}|} \hline
{\bf Scale} & {\bf Grade} \\ \hline
UNESCO & Vulnerable \\ \hline
Ethnologue (EGIDS) & 4 (Educational) \\ \hline
LEI & -- \\ \hline
Kornai & Dead \\ \hline
\end{tabular}
\caption{Scale for Naskapi}
\label{table:naskapi}
\end{table}

The {\it dead} terminology used to describe Naskapi by \citepos{kornai2013digital} metric reflects the metric being only applied to online corpora (which is minimal), and, regardless of the insensitivity of the nomenclature, it does have some merit here. When looking at the resources listed, there are no language packs for software, no Wikipedia articles, no Hunspell, no primary texts listed in OLAC, only 2415 words listed in the Cr\'ubad\'an corpus, no Indigenous tweets, no Swadesh lists, and only a brief mention in Panlex translations (90 words), and in Omniglot. Once again the data may not be perfect - this source lists the EGIDS rating at Level 5 (which I would disagree with, placing it in back in Level 4, as "The language is in vigorous use, with standardisation and literature being sustained through a widespread system of institutionally supported education.")

ODIN has exactly one IGT entry for Naskapi, from \citet{richards2004syntax}. This means that noting the translation in Example~\ref{igt1} may may double the size of ODIN's entries, although it is not the first entry in the literature (this lexeme is mentioned in \citet{macKenzie1980towards}).\footnote{This might also be transliterated as `wabush`, although it would not match the Naskapi phonological inventory. Wabush is the name of a town in Labrador, which I was told meant `hare' or `rabbit', and my pronunciation was not corrected.}

\begin{exe}
\ex
\gll wa:pus\\
hare\\
\trans `hare'
\label{igt1}
\end{exe}

Regardless of this paucity of data, there are certainly literary resources in Naskapi (see the next section) - if not many digitally. In the case of Naskapi, the Emergent level proposed by \citet{gibson2016assessing} may be more fitting than either Dead or Vital.

\subsubsection{Orthography}
Naskapi has two scripts; Latin and the Unified Canadian Aboriginal Syllabics \citep{wals-141}, which were added to Unicode in 1999.\footnote{\href{https://www.unicode.org/standard/supported.html}{https://www.unicode.org/standard/supported.html}. \last{April~27}} The Syllabics were introduced by missionaries in the 19th century, and quickly adopted by all Cree language communities, who approached near universal literacy \citep{bennett1991cree}. In Kawawachikamach and Schefferville (and on the train there), there are many examples of writing in syllabics. As well, Naskapi has its own standard orthographical conventions for Roman characters. For instance, a macron, such as \^u is used in place of a double \emph{uu} to indicate vowel length.

\citet{jancewicz2002applied} gives an insightful overview of computational technology in Naskapi. They note that Naskapi often were not involved in typesetting literature in syllabics, and that few became typists when the first syllabic typewriters were introduced. Jancewicz is particularly well placed as the author of this paper, as he and his wife were the first two missionaries sent from SIL to the Naskapi community (MacKenzie is also, as she has worked for decades with Cree communities as well as with the Naskapi). They worked with the Band Office (the local council) installing the first word processing system for syllabics, trained Naskapi speakers, and created the first Naskapi TrueType font.

Jancewicz also helped to install Keyman,\footnote{\href{https://keyman.com/}{https://keyman.com/}. \last{April~27}} "a keyboarding utility ... that allowed the programming of custom keyboard input for various languages and character sets." \citep[85]{jancewicz2002applied} Keyman is now free, open source software available on GitHub.\footnote{\href{https://github.com/keymanapp}{https://github.com/keymanapp}. \last{April~27}} It allows a user to type Roman letters which are converted to the right phrase in Syllabics, and is forgiving for phonemic variants. For instance, "ju", "chu", "tchu" and so on might all be interpreted and replaced by the appropriate syllabic \sylla{co}. % TODO Ask if this is the right syllabic
Keyman must be installed manually on each computer to use it, which reflects a considerable amount of upfront time for Jancewicz. Indeed, the importance of their support to Naskapi digital ascendancy cannot be understated (except, perhaps, by Jancewicz himself):

\begin{quote}
"Since 1988, the resident linguist has maintained all of his own language learning materials and language data on computer. He has also provided the local technical support that is needed in a small, isolated community, especially with regard to the esoteric development of computer programs that allow syllabic word processing. While it is not impossible to use computers in Native language work without a full-time, on-site computer resource person, it has been an obvious asset to have such a person available to provide training and technical support." \citep[86]{jancewicz2002applied}
\end{quote}

Currently, the school has a computer lab with over a dozen computers, but no in-house computer technician. One of the Wycliffe translators needed to visit the school to check on Keyman updates, and the students are not regularly trained in how to set up Keyman on their own, or how to set it up on their phones or other portable devices, although there have been efforts to train key teachers in how to teach computational use of Naskapi \citep{jancewicz1998developing}. While Facebook and other online platforms are increasingly popular, the majority of talking takes place in Naskapi written in local characters, or in English.

However, it is crucial that development and education regarding computational literacy continue to be mandated and improved. "Using a computer for mother-tongue language work raises speakers' assessment of the worth of their own language, as well as provides an avenue for sharing their work and ideas through reproduction and publication." \citet{jancewicz2002applied}

\citet{jancewicz2002applied} was written before wide adoption of the Unicode standard by browsers, and before the now omnipresent ubiquity of the internet and smartphones. \citet{jancewicz2012cree} gives an update on fonts available for Cree languages, including Naskapi. It also mentions Languagegeek,\footnote{\href{http://www.languagegeek.com/algon/naskapi/naskapi.html}{http://www.languagegeek.com/algon/naskapi/naskapi.html}. \last{May~3}} a website that has useful information on downloading fonts for Naskapi.

\begin{quote}
One of the most important sources for Cree Unicode fonts is the LanguageGeek website by Chris Harvey. Chris Harvey developed "Aboriginal Serif Unicode", which has gone through some changes and improvements. His current strategy is to serve logical regions of syllabic users with fonts that contain subsets of the UCAS block, rather than one font that contains them all. His work is very impressive and professional but some readers may find it difficult to read because of somewhat close letter- and especially word-spacing. \citep[17]{jancewicz2012cree}
\end{quote}

\subsubsection{Corpora creation}

In recent years, the Naskapi Development Council (NDC), which works with translators provided by the Band, has produced a Naskapi to English bilingual dictionary in three volumes \citep{MacKenzie-and-Jancewicz-1994}. The NDC is largely staffed by linguists from the Summer Institute of Linguistics, funded by Wycliffe Bible Translators and private fundraising from Christian communities.\footnote{\href{https://www.wycliffe.org/}{https://www.wycliffe.org/}. \last{April~27}} Today, the SIL linguists are a team of six: two long term linguists, and two pairs of husband and wife pairs who are training how to work as Bible translators in this community before moving on to working with other Cree communities in Canada. 

Naskapi does not have a complete Bible. A new testament, started in the 1970's, was recently published \citep{naskapi-new-testament}. Genesis, Exodus, and Psalms, have also been translated, and several children stories and books of oral legends from an elder have been produced - as well, \citet{jancewicz2002applied} note the creation of a monthly newsletter, a history, and translations of official business of the administrations (which may provide excellent multilingual corpora). The full-time translators are two people: a young woman in her mid-twenties, and an older man of around fifty years of age. At times, elders also contribute to the Bible translation effort by marking up pre-publication drafts, which they then go over with the translators.

When there is a need to come up with a new term, the elders are consulted, and they agree on an appropriate translation. For instance, {\it grill} is translated as `metal-net'. `grill' is not a  pre\"{e}xisting word in Naskapi, but `net' is, and it is easy to imagine the metaphor of a grill on which you braise meat as being a metal net. However, these decisions are not often used outside of the Bible. Likewise, when there is a term which needs to be invented at the school, the teachers there decide on an appropriate term - for instance, for situations like Halloween, where `Frankenstein' may need to be translated into a local alternative. These decisions are largely one-off, although they may be used year to year, and informally recorded in their respective domains.

The linguists use the Fieldworks Language Explorer (FLEx) \footnote{\href{https://software.sil.org/fieldworks/}{https://software.sil.org/fieldworks/}. \last{April~27}} to document new linguistic terms. FLEx was developed by SIL International, and provides linguists with an out-of-the-box solution for recording linguistics terms using interlinear glossed text. It is also open source, and available on GitHub.\footnote{\href{https://github.com/sillsdev/FieldWorks}{https://github.com/sillsdev/FieldWorks}. \last{April~27}} Users can export as a PDF (among other file formats), or export words to an online interface known as Webonary.\footnote{\href{https://www.webonary.org/configuring-the-dictionary-in-flex/}{https://www.webonary.org/configuring-the-dictionary-in-flex/}. \last{April~27}} This allows language workers to automatically create a useable, free dictionary for members of the community.

% TODO Ask if I can access the Webonary

\subsubsection{Computational tools}

There are no spell checkers, word lists, or large corpora available digitally except for the dictionary. As well as the SIL-sponsored Webonary, there is also work done by \href{http://atlas-ling.ca/}{\nolinkurl{http://atlas-ling.ca}}, which is a Canadian government-backed venture, originally cofounded by MacKenzie, who also worked on the Naskapi dictionary.\footnote{\href{http://atlas-ling.ca/}{http://atlas-ling.ca/}. \last{April~27}} This website has some options for looking at languages, but does not seem to be updated by local translators from the community. It is sourced from the previously published dictionary, which the SIL linguists have indicated is not up to date and has insufficient English to Naskapi translations. These are insufficient because of the nature of Naskapi; a root word is used with a slot system, and any word which mentions water is included under the English heading. This makes translating something as simple as "the mug is red" difficult, as one needs to know to look for `red' as a root word, and then to find the appropriate example from which you can extrapolate the correct form for translation.

There is a potentially large corpus of spoken language in Naskapi from the local radio station, but this has not been collected into a corpus. There does not appear to be any adult-level secular written corpora which could be utilised to jump-start a written corpus; \citet{jancewicz2002applied} points out that "While in some language communities it may be supposed that such an emphasis on the production of religious texts may limit the use of Native language literacy in secular community institutions, the Cree and Naskapi cultures treated in these case studies traditionally do not draw a sharp distinction between the secular and spiritual in their day-to-day life." It is also worth noting that the Band Office employs translators (who generally have other jobs - one this author talked to was a band Councilman, one of four elected officials underneath the Chief) who may be able to provide bilingual texts in English, French, or Innu.

All told, computational work that is easily accessible on the web is exceedingly limited. There are some websites in Naskapi, which could be used to make a small corpus, but there are no currently active projects working on collecting corpora for the purpose of linguistic study, and neither is there an active academic community working on Naskapi outside of the SIL translators, who may occasionally publish a paper (or, of course, a dictionary or physical book).

While FLEx is open source, none of the linguists edit the code for it or use the codebase, depending on SIL International to keep the product up to date. Keyman is likewise not edited, although it is installed on local computers. The Naskapi community website, run by the Naskapi Nation of Kawawachikamach, does have a webpage on installing syllabics,\footnote{\href{http://www.naskapi.ca/en/Install-syllabics}{http://www.naskapi.ca/en/Install-syllabics}. \last{April~27}} which may be useful for some speakers.

\subsubsection{Naskapi language resource status today}

Today, a small group of external linguists still provide much of the language resources for the community, although collaboration with the Band continues.

\begin{quote}
"The authors hope that the dichotomy between "resource people" and "aboriginal people" reflected in the section headings above would become less and less distinct. ... As a growing number of local people gain experience and expertise to become their own resource persons, such a dichotomy will dissolve, and all the vital resources for language development will exist at a community level. However, until this ideal is realised, small language communities such as these must continue to identify and avail themselves of professional and academic resources found outside their communities." \citep[89]{jancewicz2002applied}
\end{quote}

This remains just as true, fifteen years on. There is work which could be done; for instance, moving the silo'ed dictionary efforts into the public web, and using bilingual texts from the Band Office to bootstrap corpora, POS taggers, spellcheckers (there is not a Hunspell yet), and perhaps MT systems. However, this work would need to be matched by on-the-ground work by local community members - and, as \citet[90]{jancewicz2002applied} finally notes: "The initial learning curve is sometimes steep, but there is no substitute for hands-on experience."

Open source is less of a concern for Naskapi as is general software; the community is so small that any code is liable to made by community members and open sourced, anyway. However, the tools which the linguists use to develop languages benefit from open source. Any SIL missionary can contribute to FLEx, which means that incremental changes in different communities can be folded back into work in Naskapi. Likewise, any work done on Cree or any related languages can be applied or projected onto Naskapi more easily if it is open source. Open source is not a {\it sine qua non} for Naskapi technological development, then; but it could be a benefit.

% TODO Ask Bill for a copy of his MA thesis "Grammar enhanced biliteracy: Naskapi language structures for facilitating reading in Naskapi"

% \subsection{Gaelic}
% \subsection{Naskapi}
% \subsection{Kilangi} Another possible usecase?
% !TEX root = thesis.tex
\section{Methods}
\label{sec:methods}

It is customary when doing a quantitative review to give advice around best practices, to make not just the next researcher's job easier, but also to help life the quality of the state of the field, in general. In my research, I have often done the same: \citet{DBLP:journals/corr/KatzCWHVHSJCCVL15}, which came out of a workshop on sustainable software in the sciences, does a reasonable job of doing this for software citation; \citet{LittIDCC} for scientific workflows; \citet{LittEdulearn} for crowdsourcing learning materials by students in the classroom; and \citet{wiggins2013data} for public participation in science. Here, in the same vein, are some recommendations for utilising open source for LRL NLP.

\subsection{Choosing a license}
\label{choosing-a-license}

Legal advice on the internet is often preceded by the initialism IANAL, stating "I am not a lawyer", or sometimes "I am not your lawyer." The following is not meant to constitute legal advice, and I am not liable for any advice given here.

That having been said, licensing software in the open domain is definitely to been encouraged. Section~\ref{subsec:licenses} lists many licenses which are considered open source; any of them should work for most purposes (although I would recommend against the Unlicense, as it does not waive liability.) %TODO Cite?

\citet{streiter2006implementing} recommends using the GPL license for any software contributed into a software pool, their terminology for community-curated open source software. They also recommend the lesser GPL, as needed; however, GPL is preferred because it enforces that all modifications to software be brought back to the original moderator for acknowledgement, which allows for the source code to be updated. A specific example they give is of Scannell's Irish spell checker.

\begin{quote}
The case of Irish language spell checking is illustrative in this regard. Kevin Scannell developed an Irish spell checker and morphology engine in 2000, integrated it into the Ispell pool, and released everything under the GPL. Independent work at Microsoft Ireland and Trinity College Dublin led to a Microsoft-licensed Irish spell checker in 2002, but with no source code or word lists made freely available. Now, roughly five years later, the GPL tool has been updated a dozen times thanks to contributions from the community, and the data have been used directly in several advanced NLP tools, including a grammar checker and an MT system. The closed-source word list has not, to our knowledge, been updated at all since its initial release. Indeed, a version of the free word list, repackaged for use with Microsoft Word, has all but supplanted use of the Microsoft-licensed tool in the Irish-speaking community. \citep[282-283]{streiter2006implementing}
\end{quote}

I would recommend against GPL for another reason; code is often maintained by a single author, and GPL puts undo pressure on the author to maintain the code in the long term. Maintenance of code is difficult, as it involves work time that is often not paid, and as it requires that the author of the code sets expectations around levels of maintenance.

For this reason, I have always licensed my own code under the MIT license, which waives all liability and insists that the code therein is provided as-is. This makes long term maintenance easier on the maintainers, as it removes undue pressure to keep code updated. On the other hand, this leads to abandonware - code which is released into the commons and then not updated, such as TileMill which \citet{gawne2016mapmaking} used in their paper, which is no longer updated. I think that this is a reasonable price to pay for stopping burnout for the maintainers, a major factor influencing coders leaving open source. % TOOD Cite?

It is worth noting that work published without a license on a public site is not technically open source. When software is not licensed, it by default reverts (in the US legal jurisdiction, anyway) to copyright where {\it all rights are reserved}, which is by definition not FLOSS. For this reason, it is important to add a license to code if it is in your purview to do so, and if you wish to follow the open source methodology.

\subsection{Choosing repositories}
\label{choosing-repositories}

Choosing repositories is another question which needs to be answered if code is to be open sourced. All of the options mentioned so far - hosting it yourself, hosting it on an academic website, using a third-party hosting company - have their costs and benefits. If you have the resources to host the code yourself, I would suggest doing so. Unfortunately, this means that your site becomes the bottleneck for entry and discovery. Academic sites, on the other hand, may be more easily accessed by researchers in the field. However, public sites - like GitHub - are where most open source code lives, as was established in Section~\ref{subsec:where-is-open-source-code}.

For this reason, I explicitly recommend using GitHub as a storage space for open source code. Unfortunately, GitHub is a private company, and its long term goals may not align with scientists interested in century-long timelines. The Rosetta Project,\footnote{\href{https://rosettaproject.org/}{https://rosettaproject.org/}. \last{April~27}} run by the Long Now Foundation, aims to store human languages for millennia - and forward thinking on this length, while not normally used by academic researchers, raises the question of how long code ought to be stored and whether or not short term solutions are adequate.

I mentioned briefly in Section~\ref{sec:solutions} that I mirrored all of the Sourceforge repositories I found onto GitHub. Mirroring involves copying an entire code base - importantly, along with the license, so that there is no mistaking authorship - to another ecosystem or service, to maintain it in the long run. It is for this purpose that I set up the GitHub organisation @LowResourceLanguages\footnote{\href{https://github.com/lowresourcelanguages}{https://github.com/lowresourcelanguages}. \last{April~27}} (tangentially connected with the similarly named low-resource-languages repository). This organisation works as a shell to mirror code archives which might otherwise be lost.

I highly recommend mirroring all of the code that you open source, not only on GitHub, but on your personal server if you have one, and, if possible, within @LowResourceLanguages. This affords maximal accessibility, longevity, and indexing within the vibrant GitHub ecosystem.

\subsection{Sharing code without a platform}
\label{subsec:sharing-code-without-a-platform}

Of course, each of these three servers depends upon single points of failure: either your server, your provider, or your academic host. Ideally, the code would exist within large organisations to serve, as well, but there currently is no centralised codebase for linguistic code resources. OLAC, META-SHARE, LRE Maps, LingHub, LinguistList, and the LLOD all are aggregators, not hosts of code. As far as I am aware, @LowResourceLanguages on GitHub is the only code base which explicitly hosts the code. But it also relies upon GitHub's presence; which may change in ten, twenty, or a hundred years.

Peer-to-peer (p2p) technology may provide a solution to this. These work by using protocols to communicate between nodes in a network. Each node holds a copy of the file and any node which wants a copy can get it from any other node which has it. The more nodes hold a file, the easier and faster this transfer process becomes; and, if one node goes down, the other nodes can still transmit files. This allows for data permanence on a level which is unknown on on the HTTP and TCP based web.

IPFS, the InterPlanetary File System,\footnote{\href{https://ipfs.io/}{https://ipfs.io/}. \last{April~27}} is one such system which could be used to host data in the long term. The Dat project is another similar project,\footnote{\href{https://datproject.org/}{https://datproject.org/}. \last{April~27}} which has been used to save data which was deleted during by the Trump administration from US governmental websites.\footnote{\href{https://medium.com/@maxogden/project-svalbard-a-metadata-vault-for-research-data-7088239177ab}{https://medium.com/@maxogden/project-svalbard-a-metadata-vault-for-research-data-7088239177ab}. \last{April~27}} Both of these systems use hashes - deterministic DOIs based on data, which are part of the system that underly the Git tool used by GitHub and other researchers - to point to content, as opposed to locations. This allows for faster connections, offline usage with connected nodes that are not connected to the web itself, less link rot, greater specificity of content, and decentralisation.

Without going into too much detail, storing data on IPFS and then sharing it between nodes is trivial. For instance, the JSON data\footnote{\href{https://gist.github.com/RichardLitt/e60bcf9f399939b16181bf25ad6da8ba}{Available at https://gist.github.com/RichardLitt/e60bcf9f399939b16181bf25ad6da8ba}. \last{April~26}} used to analyse the low-resource-languages repository in Section~\ref{sec:solutions} could be uploaded to IPFS by installing the program and then running: {\tt ipfs add data.json}. This returns a hash (DOI) which points to the data: {\tt QmPztYpkC3aSs\-MYKDcod\-3wJtvoivbp\-NDfxNKQ6dwxnzA52}. This hash can be shared by anyone who runs IPFS, meaning that they are now storing the code on their own device, as well. It can also be accessed through a gateway to IPFS: for instance, by going to \href{http://ipfs.io/ipfs/QmPztYpkC3aSsMYKDcod3wJtvoivbpNDfxNKQ6dwxnzA52}{\nolinkurl{http://ipfs.io/ipfs/QmPztYpkC3aSsMYKDcod3wJtvoivbpNDfxNKQ6dwxnzA52}}. Uptime may depend upon the \href{https://ipfs.io}{https://ipfs.io} gateway. The code will always be available within the IPFS network for anyone who accesses it at that hash, regardless of whether the gateway is up or not. This is similar to RDF and a SPARQL gateway, except that the underpinning logic does not depend upon XML specifications, but the data itself.

There are more applications than just storing data, however. Some similar projects are already being used by non-central language communities. For instance, Guyanese communities are using p2p systems combined with GIS to map illegal logging on their land, all while being offline and not being connected to the main internet.\footnote{\href{https://www.digital-democracy.org/}{https://www.digital-democracy.org/}. \last{April~27}} \citet[90]{jancewicz2002applied} talked at length about how Naskapi development benefited from a linguist working hand-in-hand with local communities, versus long-distance arrangements as with Cree, which resulted in slower uptake of tooling and in adverse standardisation of syllabics and keymapping. A p2p network could help in these environments. It could also be used to share linguistic data within a language community, without depending upon an institutional archive in another country, a significant barrier to access and licensing control for language communities.

% \subsection{Choosing a license}
% \subsection{Choosing repositories}
% \subsection{Sharing code without a platform}
% !TEX root = thesis.tex
\section{Discussion}
\label{sec:discussion}

So: how can the open source methodology for software development low resource languages?

\subsection{Why isn't more code open?}

Finally, I'll go into a little detail on the question of why more hasn't been open sourced, and how to find open source resources.
% - Longevity of linguistic scholarship and work

\subsection{How does open source demonstrably help?}

I'll talk about use cases where open source has actually helped languages. This will include, for instance, NLTK case studies.
% \subsection{Why isn't more code open?}
% \subsection{How does open source demonstrably help?}
% !TEX root = thesis.tex
\section{Future Work}
\label{sec:future-work}

Here, I'll talk about where to go next.

\subsection{Beyond Wikipedia and Ethnologue}

I'll talk about the shortcomings of both Wikipedia as a service, and Ethnologue as a provider of language data. Specifically, I want to draw attention to how Wikipedia treats its long-term contributors, and how Ethnologue charges exorbitant fees for using its data, and what we can do to improve this.

% \subsection{An Open Data Repository}

% I'll spec out the plans for an open data repository that could be used to share data.

% - Peer-to-peer solution for sharing code
%   - Stub out example
%   - Build a web searcher for automatically getting and sharing code
%     Further Work:
%   - Open source data repositories (touch on)
%   - Working with Ethnologue


% \subsection{Storage on a p2p network}
%
% Build a web-application tool for serving a decentralized data store for endangered language tools and data
%
% Example:
%
% I have already put a subset of repositories listed on endangered-languages into IPFS, a p2p resource for storing and disseminating data in a decentralized and persistent fashion.
%
% Process:
%
% 1. `cat` the endangered-languages README.md, then `grep` for `/.*(//github\.com/.*?/[a-zA-Z0-9-]*).*/` (all github.com repos).
% 2. Output list into separate file.
% 3. `awk` the first few repos, until a random divider, and clone the git repos: `awk '1;/kuromoji-server/{exit}' ../githublist.md | xargs -n1 git clone`
% 4. `ipfs add -r repos`
% 5. `ipfs pin add repos`

% \subsection{Beyond Wikipedia and Ethnologue}
% \subsection{An Open Data Repository}
% \subsection{Storage on a p2p network}
% !TEX root = thesis.tex
\section{Conclusion}
\label{sec:conclusion}

In this thesis, I have endeavoured to show the state of low resource languages, first defining them and then looking at different metrics for judging language vitality, both on the web and offline. I have looked at what language resources are, who makes them, how they are used, and what resources are needed for low resource languages to take them from purely spoken languages to well-resourced, digitally thriving languages with a rich ecosystem of code surrounding them. I have described what open source is, and how open source can be applied to linguistic research and tooling. I mentioned the various issues surrounding funding, digital permanence, ethics, and language development in regard to LRLs.

I moved on from there to look at the state of open source code, specifically, for low resource languages, looking at the major data repositories online. I have shown a use case involving a specific NLP problem and how open source code could be applied to it. I have looked at Linked Open Data as a solution for sharing linguistic resources, and I have touched on multilingual NLP for developing on LRLs. I have looked at the state of open source code for low resource languages on GitHub, using a novel database I and others have developed to curate crowd-sourced resources. I have looked at how this tool can be used to further LRL research and NLP. 

I have examined two languages in depth, looking at the metrics applied to Scottish Gaelic and Naskapi, exploring their histories of coding and their digital presence. I have used original research I conducted in Kawawachikamach on Naskapi to help inform a new study of their digital presence, today. I have explored ways to further develop their computational and digital potential. From what I presented there, I went on to suggest licenses and repositories for future researchers in the field, and I have suggested novel ways of integrating peer-to-peer databases into language resource dissemination. I have briefly discussed what that means for LRLs, and I have outlined a half-dozen exciting areas of future research that could be undertaken in this area.

Hopefully, I have been able to impress upon the reader why open source methodologies are preferable for minority language researchers and communities. It is my belief that openness leads to better research and to better language development, and that allowing a language community to digitally ascend will enable speakers to have more opportunities and possibilities in our increasingly digital world. There is always more work to done; my hope is that, through open source licensing, we can approach this work, together.

\newpage

\bibliographystyle{apalike-refs}
\bibliography{references}

\end{document}
